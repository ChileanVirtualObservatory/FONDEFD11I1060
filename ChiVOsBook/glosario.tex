\newglossaryentry{Astronomy Data Query Language}{name={Astronomy Data Query Language}, description={Lenguaje estándar para consultar en bases de datos astronómicas, basado en \gls{sql}. La estandarización fue necesario debido a que las variantes más populares de SQL, ya sea comerciales o de código abierto, difieren ligeramente entre ellas, además que IVOA requería de una manera estándar de especificar una región, que SQL no la tenía}, text={\emph{Astronomy Data Query Language}}}
\newglossaryentry{Simple Cone Search Protocol}{name={Simple Cone Search Protocol}, description={Protocolo que define una consulta simple para obtener registros desde un catálogo de fuentes astronómicas. La consulta describe una posición en el cielo y una distancia angular, definiendo un cono en el cielo. La respuesta retorna una lista de fuentes astronómicas desde los catálogos cuyas posiciones están dentro del cono, en formato \texttt{VOTable}. \url{http://ivoa.net/Documents/latest/ConeSearch.html}}, text={\emph{Simple Cone Search Protocol}}}
\newglossaryentry{Simple Image Access Protocol}{name={Simple Image Access Protocol}, description={Esta especificación define un protocolo para la obtención de datos de imágenes desde una variedad de repositorios de imágenes astronómicas, a trav\'es de una interfaz uniforme. La interfaz está destinada a ser razonablemente fácil de implementar por los proveedores de servicios. La consulta define una región rectangular en el cielo, la cual es usada para encontrar imágenes candidatas. El servicio retorna una lista de imágenes en formato \texttt{VOTable}. Por cada imagen candidata se entrega una URL de referencia, la cual permite acceder a la imagen. La imagen puede ser retornada en varios formatos gráficos (\gls{fits}, JPEG, etc). \url{http://ivoa.net/Documents/SIA/}}, text={\emph{Simple Image Access Protocol}}}
\newglossaryentry{Simple Spectral Access Protocol}{name={Simple Spectral Access Protocol}, description={Este protocolo define una interfaz uniforme para descubrir y acceder remotamente a espectros de una dimensión. Se basa en un modelo de datos más general, capaz de describir los datos espectrofotom\'etricos, incluyendo series de tiempo y las distribuciones espectrales de energía (SED), así como 1-D de espectros. Los conjuntos de datos de candidatos disponibles se describen de manera uniforme en un documento de formato de \texttt{VOTable} que se devuelve en la respuesta a la consulta. \url{http://ivoa.net/documents/SSA/}}, text={\emph{Simple Spectral Access Protocol}}}
\newglossaryentry{Table Access Protocol}{name={Table Access Protocol}, description={Este protocolo define un servicio general de acceso a datos de tablas, incluyendo catálogos astronómicos, como tambi\'en base de datos generales de tablas. El acceso se provee tanto a la base de datos y a la tabla de metadatos para la tabla actual. La versión actual del protocolo incluye soporte para consultas en múltiples lenguajes, incluyendo consultas especificadas en \gls{adql} y \gls{pql}. Tambi\'en incluye soporte para consultas sincrónicas y asíncronas. Este servicio está ligado netamente al modelo de datos que se usará. \url{http://www.ivoa.net/documents/TAP/}}, text={\emph{Table Access Protocol}}}
\newglossaryentry{semantics}{name={Semantics}, description={El grupo de semánticas de \gls{ivoa} explora las tecnologías en el área de semánticas, con el objetivo de producir nuevos estándares que ayuden la interoperabilidad de los sistemas del VO. Este grupo está enfocado en el significado o la interpretación de las palabras, frases u otras formas de lenguaje en el contexto de la astronomía. Esto incluye la descripcón estándar de objetos astrofísicos, tipos de datos, conceptos, eventos, o algún otro tipo de fenómeno en astronomía. Este grupo estudia la relación entre palabras, símbolos y conceptos, tanto como el significado de esa representación, como por ejemplo, ontologías. Este grupo cubre el lenguaje natural en astronomía, incluyendo consultas, traducciones, e internacionalización de interfaces. \url{http://ivoa.net/twiki/bin/view/IVOA/IvoaSemantics}}, text={\emph{semantics}}}
\newglossaryentry{capability}{name={Capability}, description={Lo que un servicio puede hacer por usted. Por ejemplo, si se ofrece un servicio de \gls{scs}, o un servicio de \gls{tap}, o una interfaz web, etc. Un servicio puede ofrecer más de una \emph{capacidad}}, text={\emph{capability}}}
\newglossaryentry{crossmatch}{name={Crossmatch}, description={Corresponde al proceso de encontrar registros en una colección de datos que coincidan con los de otro. Por ejemplo, que el ID de un objeto en un catálogo óptico sea el mismo que para una lista de fuentes de rayos X. Claramente esto requiere que ambos servicios de datos sean interoperables}, text={\emph{crossmatch}}}
\newglossaryentry{Data Model}{name={Data Model}, description={Corresponde a una estructura lógica y estandarizada para una conjunto de datos, lo que hace posible que las herramientas le den sentido a los datos devueltos. Esto no es lo mismo que un formato para los datos. Un modelo de datos, por ejemplo, debería obligar a que haya un conjunto de datos que contenga una serie de registros que representen a objetos, y que cada registro debiese contener cantidades para representar \gls{ra}, \gls{dec}, \'epoca (\emph{epoch}) de observación, entre otros. Por otro lado, un formato de datos especifica que el archivo es una tabla, que las entradas están en ASCII, que los enteros son de 16 \texttt{bit}, etc}, text={\emph{data model}}}
\newglossaryentry{Flexible Image Transport System}{name={Flexible Image Transport System}, description={Formato estándar para imágenes astronómicas y tablas. Los FITS tambi\'en especifican una manera estándar para expresar metadatos en duplas valor-palabra clave, y especifica un número pequeño de palabras claves obligatorias, y un número un poco mayor de palabras claves reservadas, las cuales tienen que tener una significado en particular si un archivo los usa. Cada tabla de FITS puede ser expresada como un \gls{votable}, pero no necesariamente \emph{vice versa}, ya que el formato VOTable puede contener \gls{metadata} más complicada}, text={\emph{Flexible Image Transport System}}}
\newglossaryentry{International Virtual Observatory Alliance}{name={International Virtual Observatory Alliance}, description={Organización que fomenta la colaboración entre \gls{vo} de todo el mundo, acordando los estándares necesarios para hacer que los VO trabajen}, text={\emph{International Virtual Observatory Alliance}}}
\newglossaryentry{metadata}{name={Metadata}, description={Datos sobre los datos. Normalmente utilizada para describir información estandarizada que va con un archivo de datos (e.g. palabras claves de un FITS), o información de un registro de entrada que describe un recurso como un servicio de datos, o información que describe la estructura de una base de datos: una lista de tablas, los nombres de sus columnas y sus \gls{ucd}}, text={\emph{metadata}}}
\newglossaryentry{NASA/IPAC Extragalactic Database}{name={\gls{nasa}/\gls{ipac} Extragalactic Database}, description={Es una gran base de datos de objetos astronómicos mantenida por IPAC en Caltech. Consultable a trav\'es de una interfaz web y tambi\'en como un \gls{scs}}, text={NASA/IPAC Extragalactic Database}}
\newglossaryentry{voportal}{name={VO Portal}, description={Página web que ofrece un estilo de ``ventanilla única'' de acceso a las herramientas y servicios \gls{vo}, en contraposición a herramientas diferentes e independientes}, text={\emph{VO Portal}}}
\newglossaryentry{registry}{name={Registry}, description={Las páginas amarillas del \gls{vo}. Cualquier recurso (usualmente, pero no siempre, un servicio de algún tipo) tendrá un registro de entrada con información estandarizada que describe qu\'e es, cómo acceder a \'el, entre otra información. Hay muchos registros en operación, pero se actualizan unos de otros. Las herramientas encuentran servicios y acceden a sus datos a trav\'es de uno o más registros}, text={\emph{registry}}}
\newglossaryentry{service}{name={Service}, description={Algo en internet que hará activamente algo por usted, en contraposición a lo que es un pasivo repositorio de información. Por ejemplo, un servicio de imágenes puede tener un gran atlas de imágenes, pero tambi\'en ofrece la posibilidad de realizar consultas para obtener una imagen recortada desde un parte del cielo en particular}, text={\emph{service}}}
\newglossaryentry{samp}{name={SAMP}, description={Protocolo de mensajería que permite a las aplicaciones intercambiar mensajes y datos entre ellas a trav\'es de un \emph{hub} central, el cual corresponde a una pequeña aplicación que corre en la estación de trabajo del usuario. Por ejemplo, se puede recuperar un catálogo desde un servicio \gls{scs} usando \text{TOPCAT}, pero además se tiene corriendo herramientas como \texttt{Aladin} o \texttt{DS9}, mirando una imagen de la misma parte del cielo. \texttt{TOPCAT} puede entonces enviar la tabla a la herramienta de imágenes, la cual puede superponer los objetos sobre la imagen}, text={SAMP}}
\newglossaryentry{simbad}{name={Simbad}, description={Una gran base de datos, compatible con los \gls{vo}, de objetos astronómicos mantenido por el \gls{cds}. Puede ser consultada de diferentes maneras, incluyendo formularios web con patrones fijos, o de manera completamente flexible mediante \gls{adql}, utilizando el estándar \gls{tap}}, text={\emph{Simbad}}}
\newglossaryentry{singlesignon}{name={Single sign on}, description={Muchos conjuntos de datos en un \gls{vo} son públicos, pero muchos tienen algunas restricciones de propiedad, que necesitan de un usuario para saber quienes son, y para que el servicio funcione si se le es permitido el acceso. La idea detrás de \emph{single sing on} es que sólo se debiese autenticar una vez por sesión. La infraestructura t\'ecnica para esto ya se encuentra aprobada, pero aún no está totalmente implementada}, text={single sign-on}}
\newglossaryentry{synchservices}{name={Synchronous Services}, description={Es aquel donde el usuario necesita permanecer conectado a un servicio remoto e interactuar con \'el en tiempo real}, text={servicio sincrónico}}
\newglossaryentry{asynchservices}{name={Asynchronous Services}, description={Es aquel donde el usuario especifica un trabajo a ser hecho, desconectándose y obteniendo el resultado despu\'es}, text={servicio asincrónico}}
\newglossaryentry{vizier}{name={Vizier}, description={Gran colección de catálogos astronómicos y tablas almacenadas en \gls{cds}. Además de los ya conocidos catálogos, incluye muchas tablas publicadas como parte de artículos científicos astronómicos. Los catálogos y tablas de \emph{Vizier} pueden ser consultados a trav\'es de las páginas web de CDS, o a trav\'es de diferentes herramientas \gls{vo}}, text={\emph{Vizier}}}
\newglossaryentry{voevent}{name={VO Event}, description={Formato estándar para describir un evento astronómico, tal como una supernova o una explosión de rayos gama. Un paquete \texttt{VOEvent} contiene información sobre la localización, tiempo, y la probable naturaleza de un fugaz evento astronómico, así como información sobre su observación}, text={\texttt{VOEvent}}}
\newglossaryentry{vospace}{name={VO Space}, description={Almacenamiento remoto bajo estándares \gls{vo}. Como muchos otros estándares VO, \texttt{VOSpace} no especifica como el almacenamiento remoto es organizado, sino que cómo una interfaz de una herramienta debe hacerlo, por lo que pareciera lo mismo desde el punto de vista del usuario}, text={\texttt{VOSpace}}}
\newglossaryentry{votable}{name={VO Table}, description={Formato estándar para representar e intercambiar datos tabulados dentro del \gls{vo}. La mayoría de los archivos de datos ahora ofrecen exportación de datos en formato \texttt{VOTable}, y un gran número de herramientas pueden leer \texttt{VOTable}. Cualquier tabla de \texttt{FITS} puede tambi\'en ser expresada como como un \texttt{VOTable}, pero siempre al rev\'es, debido a que un \texttt{VOTable} puede contener más \gls{metadata} estructurada, y no sólo pares palabra-valor}, text={\texttt{VOTable}}}
\newglossaryentry{Unified Content Descriptor}{name={Unified Content Descriptor}, description={Vocabulario estándar para describir cantidades de datos astronómicos. No se especifica el nombre de una cantidad, o su unidad, si no más bien qu\'e tipo de cantidad es. Por ejemplo, una columna en una tabla podría tener el nombre ``T-kin'' y un UCD ``phys.temperature'', que establece que es una temperatura, pero que no implica una unidad en particular}, text={Unified Content Descriptor}}
\newglossaryentry{units}{name={Units}, description={Cadena (\emph{string}) estandarizada de \gls{ivoa} para especificar las unidades de una cantidad}, text={\emph{units}}}
\newglossaryentry{j2000}{name={J2000}, description={Corresponde a la fecha juliana 2451545.0 \gls{tt}, lo que es equivalente a: 12 horas del 1 de enero de 2000 TT; 1 de enero de 2000 a las 11:59:27.816 \gls{tai}; y 1 de enero de 2000 a las 11:58:55.816 \gls{utc}}, text={J2000}}
\newglossaryentry{b1950}{name={B1950}, description={Corresponde al inicio del \gls{anoBesseliano} de 1950}, text={B1950}}
\newglossaryentry{anoBesseliano}{name={Año besseliano}, description={En honor al matemático y astrónomo alemán Friedrich Bessel, un año besseliano se define como el comienzo de un año en el cual el momento de longitud media del sol, incluido el efecto de aberración de la luz y medido desde el equinoccio medio de la fecha, es exactamente 280 grados}, text={año besseliano}}
\newglossaryentry{sesame}{name={Sesame}, description={Resolvedor de nombres que retorna, a partir de una cadena (\emph{string}) que representa la designación de un objeto astronómico ubicado fuera del Sistema Solar, la posición de tal objeto en el cielo y algunos otros pocos detalles. Opera consultando las siguientes bases de datos: \gls{simbad}, \gls{ned} y \gls{vizier}}, text={\texttt{Sesame}}}
%\newglossaryentry{}{name={}, description={}, text={}}
%\newglossaryentry{}{name={}, description={}, text={}}
%\newglossaryentry{}{name={}, description={}, text={}}
%\newglossaryentry{}{name={}, description={}, text={}}
%\newglossaryentry{}{name={}, description={}, text={}}
%\newglossaryentry{}{name={}, description={}, text={}}
%\newglossaryentry{}{name={}, description={}, text={}}
%\newglossaryentry{}{name={}, description={}, text={}}
%\newglossaryentry{}{name={}, description={}, text={}}
%\newglossaryentry{}{name={}, description={}, text={}}
%\newglossaryentry{}{name={}, description={}, text={}}
%\newglossaryentry{}{name={}, description={}, text={}}
%\newglossaryentry{}{name={}, description={}, text={}}
%\newglossaryentry{}{name={}, description={}, text={}}
%\newglossaryentry{}{name={}, description={}, text={}}
%\newglossaryentry{}{name={}, description={}, text={}}
%\newglossaryentry{}{name={}, description={}, text={}}
%\newglossaryentry{}{name={}, description={}, text={}}
%\newglossaryentry{}{name={}, description={}, text={}}
%\newglossaryentry{}{name={}, description={}, text={}}
%\newglossaryentry{}{name={}, description={}, text={}}
%\newglossaryentry{}{name={}, description={}, text={}}
%\newglossaryentry{}{name={}, description={}, text={}}
%\newglossaryentry{}{name={}, description={}, text={}}

