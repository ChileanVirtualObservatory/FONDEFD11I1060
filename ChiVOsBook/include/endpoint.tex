\chapter{Capa de aplicación}

La definición de requerimientos de datos involucra en particular a un área de arquitectura de software recomendada por \gls{ivoa}. Esta capa es la de estándares y protocolos de acceso a los datos (\emph{Data Access Protocol Standards}).

Los datos y metadatos astronómicos están siendo distribuidos mediante distintos mecanismos que le permitan a la comunidad astronómica y científica acceder a ellos de forma fácil y estandarizada. Por lo general, estos accesos son mediante páginas web, aplicaciones de escritorio, servicios \gls{ftp}, etc. La colaboración de distintos observatorios virtuales ha permitido crear protocolos estandarizados para consultar y acceder a datos y metadatos de distintas fuentes de información. Estos protocolos fueron creados para permitir que la implementación fuese más fácil.

Durante la ejecución del proyecto se ha estudiado estos protocolos, enmarcado en la idea de \gls{vo}, y la arquitectura que se debe respetar.

