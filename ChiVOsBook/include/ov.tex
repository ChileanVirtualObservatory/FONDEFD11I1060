\section*{Introducción}

Desde el año 2002, proyectos de Observatorios Virtuales (VO’s, por sus siglas en ingl\'es) comenzaron a integrar la Alianza Internacional de Observatorios Virtuales bajo el \emph{Guidelines for Participation}\footnotemark{}.

\footnotetext{La documentación se puede encontrar en \url{http://www.ivoa.net/documents/latest/IVOAParticipation.html}.}

Esos proyectos fueron fundados bajo programas privados y gubernamentales nacionales e internacionales en colaboración con centros de estudios científicos, universidades y otros. Quienes integran este proyecto, el Observatorio Virtual, comparten conocimientos entre ellos y la comunidad de modo estandarizado, siendo ellos mismos quienes desarrollan estos estándares para el intercambio de información e interoperabilidad.

La Tab.~\ref{tab:memivoa} muestra los miembros de \gls{ivoa} hasta abril de 2015\footnotemark.

\footnotetext{\url{http://www.ivoa.net/about/member-organizations.html}.}

\begin{table}[ht!]
	\centering
	\begin{tabular}{c|p{2.3in}|p{2.5in}}
		Proyecto & Nombre & Dirección \\\hline\hline
		NOVA & Nuevo Observatorio Virtual Argentino & \url{http://nova.conicet.gov.ar/} \\\hline
		ArVO & Armenian Virtual Observatory & \url{http://www.aras.am/Arvo/arvo.htm} \\\hline
		AstroGrid & UK's Virtual Observatory & \url{http://www.astrogrid.org/} \\\hline
		Aus-VO & Australian Virtual Observatory & \url{http://aus-vo.org.au/} \\\hline
		BRAVO & Brazilian Virtual Observatory & \url{http://bravo.iag.usp.br/} \\\hline
		CVO & Canadian Virtual Observatory & \url{http://www.cadc-ccda.hia-iha.nrc-cnrc.gc.ca/cvo/} \\\hline
		ChiVO & Chilean Virtual Observatory & \url{http://www.chivo.cl/} \\\hline
		China-VO & Chinese Virtual Observatory & \url{http://www.china-vo.org/} \\\hline
		ESA-VO & European Space Agency Virtual Observatory & \url{http://www.sciops.esa.int/index.php?project=SAT\&page=ESAVOIntro} \\\hline
		EURO-VO & European Virtual Observatory & \url{http://www.euro-vo.org/} \\\hline
		GAVO & German Astrophysical Virtual Observatory & \url{http://www.g-vo.org/} \\\hline
		HVO & Hungarian Virtual Observatory & \url{http://hvo.elte.hu/en/} \\\hline
		JVO & Japanese Virtual Observatory & \url{http://jvo.nao.ac.jp/} \\\hline
		OV-France & Observatoires Virtuels France & \url{http://www.france-vo.org/} \\\hline
		RVO & Russian Virtual Observatory & \url{http://www.inasan.rssi.ru/eng/rvo/} \\\hline
		SA\textsuperscript{3} & South African Astroinformatics Alliance & \url{http://www.sa3.ac.za/} \\\hline
		SVO & Spanish Virtual Observatory & \url{http://svo.cab.inta-csic.es/main/index.php} \\\hline
		VObs.it & Italian Virtual Observatory & \url{http://vobs.astro.it/} \\\hline
		UKR VO & Ukrainian Virtual Observatory & \url{http://www.ukr-vo.org/} \\\hline
		VAO & US Virtual Astronomical Observatory & \url{http://www.usvao.org/} \\\hline
		VO-I & Virtual Observatory India & \url{http://vo.iucaa.ernet.in/~voi/} \\
	\end{tabular}
	\caption{Integrantes de \gls{ivoa}.}
	\label{tab:memivoa}
\end{table}

Con la incorporación de \gls{chivo} a la lista de organizaciones miembro de \gls{ivoa}, el continente americano se equipara en cantidad de VO's al continente asiático\footnotemark{}, siendo el europeo el que mantiene el liderato. La incorporación de Chile a este selecto grupo no hace más que consolidar la postura a nivel informático del país con mayor presencia de observatorios astronómicos a nivel mundial.

\footnotetext{Esto es considerando a Armenia y Rusia como miembros asiáticos. Si se les considera como europeos, Am\'erica es el segundo continente con mayor cantidad de VO's.}

Considerando todo lo anterior, al comienzo del proyecto se detalló un listado de requerimientos, los cuales se repasan a continuación:

\begin{description}
	\item[Buscar por coordenadas o región en el cielo] Se podrá realizar búsquedas de posición mediante coordenadas y radio angular (cónicas) o por	región del cielo. Los parámetros de las coordenadas pueden ser en distintos sistemas como ecuatorial, eclíptico, galáctico o supergaláctico. Los parámetros ingresados se convertirán al sistema de la fuente de datos, para así poder realizar las búsquedas, \gls{ivoa} utiliza los sistemas de coordenadas ICRS y ecuatoriales \gls{j2000}. En un principio, el sistema ofrecerá servicio de búsqueda por coordenadas cónicas y más adelante, en caso de ser vía un portal web, se ofrecerá búsqueda por región del cielo. El sistema tambi\'en deberá permitir buscar simultáneamente un listado de coordenadas.
	\item[Buscar por nombre o tipo de objeto] El sistema deberá permitir buscar por nombres de objetos que se encuentren definidos en Sesame, del \emph{Centre de Donn\'ees astronomiques de Strasbourg} (CDS). Por otro lado, la búsqueda por tipo o subtipo de objeto, tales como estrellas en formación, estrellas nebulosas planetarias, supernovas, galaxias, cometas, entre otros, permitirá al usuario encontrar datos relacionados con una problemática en especial. En un principio, el sistema realizará estas búsquedas acorde a la información presente en los catálogos. A futuro, el sistema deberá permitir minería de datos para detección de tipos de objetos similares, esto es posible debido a que existen clasificaciones discretas que permiten clasificar los objetos que se encuentran en las observaciones. Este tipo de búsquedas, se transforman en búsquedas por coordenadas, ya que al buscar por un nombre, por ejemplo, Sesame responde con la correspondiente ubicación del objeto en coordenadas. Y luego de ello se procede a realizar la búsqueda por coordenadas correspondiente. El resultado de la búsqueda deberá facilitar la obtención de datos para ser analizados como secuencias de tiempo.
	\item[Buscar por \glslink{metadata}{metadatos} espectrales] Se podrá realizar búsquedas por \glslink{metadata}{metadatos} espectrales, lo cual consiste en búsqueda por banda o rango de frecuencia, búsquedas por líneas espectrales y corrimiento al rojo o búsquedas por resolución espectral. Específicamente, hay dos enfoques:
		\begin{itemize}
			\item Galáctico: Por frecuencia en reposo y velocidad radial.
			\item Extragaláctico: Por frecuencia en reposo y corrimiento al rojo.
		\end{itemize}
		La frecuencia en reposo incluye búsquedas por mol\'ecula, transición de mol\'ecula (vibracional, rotacional o electrónica) o frecuencia de línea espectral.
	\item[Buscar por \glslink{metadata}{metadatos} espaciales] Se podrá realizar búsquedas espaciales en base a parámetros relacionados con rangos de resolución angular y campos de visión y siempre en base a coordenadas. Los parámetros de las coordenadas pueden ser en distintos sistemas como ecuatorial, eclíptico, galáctico o supergaláctico. Los parámetros ingresados se convertirán al sistema de la fuente de datos, para así poder realizar las búsquedas. Además, se podrá especificar parámetros relacionados con la forma de las observaciones (rectangulares o redondas).
	\item[Buscar por \glslink{metadata}{metadatos} temporales] Se podrá realizar búsquedas por \glslink{metadata}{metadatos} temporales que pueden ser clasificadas en dos tipos de	búsquedas:
		\begin{itemize}
			\item Cuando fue realizada la observación, incluyendo cuantas veces se observó un objeto y/o el	intervalo entre observaciones.
			\item Nivel de ruido, duración de la observación o tiempo de integración. Dado un ruido, se necesita un tiempo de integración que depende de la frecuencia observada y del clima.
		\end{itemize}
	\item[Buscar por polarización] Cada imagen se puede dividir en cuatro parámetros llamados los parámetros de Stokes o en dos: izquierda y derecha. En radio astronomía no suele hacerse debido a que requiere una alta precisión	del instrumento, en el caso de ALMA se requiere que est\'e lista la calibración. El sistema debe permitir buscar si existe o no polarización en alguno de los parámetros de Stokes: I, Q, U o V.
	\item[Cruzamiento de información] La búsqueda cruzada debe permitir al usuario realizar las búsquedas mencionadas anteriormente	en múltiples fuentes de datos distribuidos globalmente, sin importar su tipo. Lo que permitirá obtener todos los datos existentes sobre un objeto o área espacial y así evitar realizar observaciones innecesarias debido al descubrimiento de observaciones existentes. Los tipos de fuentes pueden ser Sesame, ALMA u Observatorios Virtuales, que cumplan con los estándares de \gls{ivoa}. La búsqueda de un mismo objeto en distintas fuentes de información, debido a que en cada fuente el instrumento tiene un margen de error en cuanto a la posición del objeto, debe ser capaz de realizar	una intersección entre los radios de margen de error de las distintas fuentes para identificar al objeto en una búsqueda cruzada.
	\item[Simulaciones] Es a veces necesario realizar comparaciones con observaciones obtenidas a trav\'es de simulaciones,	así cómo es costoso realizar una observación dos veces, para una simulación puede ser aún más, debido a que dependiendo de la magnitud, hay simulaciones que requieren una gran capacidad de cómputo.
	\item[Servicios Bibliográficos] Las herramientas existentes cumplen su función correctamente. Por ello, basta con que al buscar un objeto, se despliegue tambi\'en resultados de investigaciones que se hayan realizado al respecto con	un enlace a SIMBAD o similares.
\end{description}

A partir de estos requerimientos, surge un listado de casos de uso, los cuales fueron formulados por los siguientes astrónomos:

\begin{itemize}
	\item Diego Mardones, Universidad de Chile.
	\item Lars Nyman, \gls{alma}.
	\item Neil Nagar, Universidad de Concepción.
	\item Nelson Padilla, Pontificie Universidad Católica. 
	\item Juan de Santander, Instituto de Astrofísica de Andalucía, España.
	\item Amelia Bayo, \gls{eso}.
\end{itemize}

Antes de detallas los casos de uso, es importante destacar algunos aspectos de relevancia que sirven como base para entender el trabajo realizado:

\begin{itemize}
	\item Los datos astronómicos en sus diferentes formatos se dividen en: Metadata y Binarios. La metadata son datos que describen los datos de la observación, y los binarios son los datos obtenidos de la observación.
	\item En la primera etapa el proyecto trabajará con archivos FITS.
	\item Las búsquedas en el sistema se realizan sobre la metadata. El resultado de una búsqueda es un set de datos candidatos que coinciden con los parámetros fijados en la búsqueda.
	\item La interacción final con el sistema es cuando el usuario decide qué datos necesita, los selecciona y los descarga de forma local.
\end{itemize}

\subsection*{Casos de Uso Generales}

Los siguientes casos de usos involucran de manera general a varios requerimientos, por lo que se presentan en una categoría especial. 

\paragraph{Caso de uso \#1:}

\begin{description}[noitemsep]
	\item[Objetivo] Filtrar los resultados de la búsqueda en el portal web.
        \item[Actor] Usuario.
        \item[Necesidad] Esencial.
        \item[Prioridad] Alta.
        \item[Requerimientos Referenciados] del 1 al 7.
        \item[Descripción] Ya que el usuario puede recibir una gran cantidad de resultados, éste debe ser capaz de realizar un filtro sobre alguna columna de los datos recibidos al realizar la búsqueda para así poder identificar los datos que se encuentre buscando. Este filtro puede ser realizado en el mismo navegador del usuario o a través de una nueva consulta. 
\end{description}

\paragraph{Caso de uso \#2:}

\begin{description}[noitemsep]
        \item[Objetivo] Descargar datos desde el portal web.
        \item[Actor] Usuario.
        \item[Necesidad] Esencial.
        \item[Prioridad] Alta.
        \item[Requerimientos Referenciados] del 1 al 7.
        \item[Descripción] Una vez que el usuario encuentre los datos que requiere, éste procede a descargarlos. Para ello el sistema provee un enlace directo a la fuente de los datos.
\end{description}

\paragraph{Caso de uso \#3:}

\begin{description}[noitemsep]
	\item[Objetivo] Visualizar una representación gráfica que compare los \glslink{metadata}{metadatos} de los resultados de la búsqueda.
        \item[Actor] Usuario.
        \item[Necesidad] Esencial.
        \item[Prioridad]  Media.
        \item[Requerimientos Referenciados] del 1 al 7.
	\item[Descripción] Una vez que el usuario reciba un conjunto de resultados, podrá visualizarlos en gráficos acordes al tipo de búsqueda y en base a los \glslink{metadata}{metadatos} recibidos por la búsqueda.
\end{description}

\paragraph{Caso de uso \#4:}

\begin{description}[noitemsep]
	\item[Objetivo] Visualizar un producto o subproducto de los datos de la observación presente en los \glslink{metadata}{metadatos}.
        \item[Actor] Usuario.
        \item[Necesidad] Esencial.
        \item[Prioridad] Alta.
        \item[Requerimientos Referenciados] del 1 al 7.
        \item[Descripción] Visualización de la forma geométrica de las observaciones (rectangulares o redondas); cobertura UV; calibración de paso de banda; espectro observado; imagen observada en un plano del producto cuando sea posible (podría ser parte de análisis en vez de observación).
\end{description}

\paragraph{Caso de uso \#5:}

\begin{description}[noitemsep]
	\item[Objetivo] Análisis de cubo \gls{fits}.
        \item[Actor] Usuario.
        \item[Necesidad] Esencial.
        \item[Prioridad] Alta.
        \item[Requerimientos Referenciados] del 1 al 7.
        \item[Descripción] Selecciona pixeles; proyección en una dimensión; flujo en un área; series de tiempo.
\end{description}

\paragraph{Caso de uso \#6:}

\begin{description}[noitemsep]
	\item[Objetivo] Análisis de \gls{asdm}.
        \item[Actor] Usuario.
        \item[Necesidad] Deseable.
        \item[Prioridad] Media.
        \item[Requerimientos Referenciados] del 1 al 7.
        \item[Descripción] Selecciona voxeles; proyección en una dimensión; flujo en un área; series de tiempo.
\end{description}

\paragraph{Caso de uso \#7:}

\begin{description}[noitemsep]
        \item[Objetivo] Los resultados de la búsqueda deben ser analizables para secuencias de tiempo.
        \item[Actor] Usuario.
        \item[Necesidad] Esencial.
        \item[Prioridad] Baja.
        \item[Requerimientos Referenciados] del 1 al 7.
        \item[Descripción] El usuario luego de realizar una búsqueda por tipo y/o subtipo de objeto, puede obtener los resultados de la búsqueda para ser analizados como secuencias de tiempo.
\end{description}


\subsection*{Buscar por coordenada o región del cielo}

\paragraph{Caso de uso \#8:}

\begin{description}[noitemsep]
        \item[Objetivo] Ingresar al portal web y realizar una búsqueda por coordenadas.
        \item[Actor] Usuario.
        \item[Necesidad] Esencial.
        \item[Prioridad] Alta.
        \item[Requerimientos Referenciados] 1.
	\item[Descripción] El usuario ingresa al Portal Web, rellena los campos de coordenadas y radio angular o región de cielo y realiza una búsqueda. Los parámetros de coordenada pueden pertenecer al sistema ecuatorial (\gls{j2000} o \gls{b1950}), eclíptico, galáctico o supergaláctico.
\end{description}

\paragraph{Caso de uso \#9:}

\begin{description}[noitemsep]
        \item[Objetivo] Realizar una búsqueda de listado de coordenadas.
        \item[Actor] Usuario.
        \item[Necesidad] Esencial.
        \item[Prioridad] Alta.
        \item[Requerimientos Referenciados] 1.
        \item[Descripción] El usuario ingresa al Portal Web, ingresa una lista de coordenadas y radios angulares o regiones de cielo y realiza una búsqueda. Los parámetros de coordenada pueden pertenecer al sistema ecuatorial (J2000 o B1950), eclíptico, galáctico o supergaláctico. También el usuario puede subir un archivo con un formato establecido por el sitio con el listado de coordenadas.
\end{description}


\subsection*{Buscar por nombre o tipo de objeto}

\paragraph{Caso de uso \#10:}

\begin{description}[noitemsep]
        \item[Objetivo] Ingresar al Portal Web y realizar una búsqueda por nombre en base a Sesame.
        \item[Actor] Usuario.
        \item[Necesidad] Esencial.
        \item[Prioridad] Alta.
        \item[Requerimientos Referenciados] 2.
        \item[Descripción] El usuario ingresa al Portal Web, rellena el campo de nombre según los nombres definidos en Sesame y realiza una búsqueda.
\end{description}

\paragraph{Caso de uso \#11:}

\begin{description}[noitemsep]
        \item[Objetivo] Ingresar al Portal Web y realizar una búsqueda por nombre en base a catálogo de un instrumento.
        \item[Actor] Usuario.
        \item[Necesidad] Esencial.
        \item[Prioridad] Alta.
        \item[Requerimientos Referenciados] 2.
        \item[Descripción] El usuario ingresa al Portal Web, rellena el campo de nombre según los nombres definidos en Catálogos específicos de un instrumento y realiza una búsqueda.
\end{description}

\paragraph{Caso de uso \#12:}

\begin{description}[noitemsep]
        \item[Objetivo] Ingresar al Portal Web y realizar una búsqueda por tipo de objeto en un área del cielo.
        \item[Actor] Usuario.
        \item[Necesidad] Esencial.
        \item[Prioridad] Alta.
        \item[Requerimientos Referenciados] 2.
        \item[Descripción] El usuario ingresa al Portal Web, rellena los campos de tipo y/o subtipos de objetos y un área del cielo y realiza una búsqueda.
\end{description}


\subsection*{Buscar por metadatos espectrales}

\paragraph{Caso de uso \#13:}

\begin{description}[noitemsep]
        \item[Objetivo] Ingresar al Portal Web y realizar una búsqueda espectral extragaláctica.
        \item[Actor] Usuario.
        \item[Necesidad] Esencial.
        \item[Prioridad] Alta.
        \item[Requerimientos Referenciados] 3.
        \item[Descripción] El usuario ingresa al Portal Web, rellena los campos de banda o rango de frecuencia;  y/o líneas espectrales y corrimiento al rojo (z); y/o resolución espectral y/o ruido y realiza una búsqueda.
\end{description}

\paragraph{Caso de uso \#14:}

\begin{description}[noitemsep]
        \item[Objetivo] Ingresar al Portal Web y realizar una búsqueda espectral galáctica.
        \item[Actor] Usuario.
        \item[Necesidad] Esencial.
        \item[Prioridad] Alta.
        \item[Requerimientos Referenciados] 3.
        \item[Descripción] El usuario ingresa al Portal Web, rellena los campos de banda o rango de frecuencia;  y/o líneas espectrales y velocidad radial (v\_r); y/o resolución espectral y/o ruido y realiza una búsqueda.  Línea espectral incluye campos de moléculas, transición de moléculas (vibracional, rotacional o electrónica) o frecuencia en reposo de la línea espectral.
\end{description}

\paragraph{Caso de uso \#15:}

\begin{description}[noitemsep]
        \item[Objetivo] Realizar una búsqueda de listado de frecuencias.
        \item[Actor] Usuario.
        \item[Necesidad] Esencial.
        \item[Prioridad] Alta.
        \item[Requerimientos Referenciados] 3.
        \item[Descripción] El usuario ingresa al Portal Web, ingresa una lista de frecuencias y realiza una búsqueda. También el usuario puede subir un archivo con un formato establecido por el sitio con el listado de frecuencias.
\end{description}


\subsection*{Buscar por metadatos espaciales}

\paragraph{Caso de uso \#16:}

\begin{description}[noitemsep]
        \item[Objetivo] Ingresar al Portal Web y realizar una búsqueda por resolución angular y/o campos de visión.
        \item[Actor] Usuario.
        \item[Necesidad] Esencial.
        \item[Prioridad] Media.
        \item[Requerimientos Referenciados] 4.
        \item[Descripción] El usuario ingresa al Portal Web, rellena los campos de resolución angular y/o campos de visión y realiza una búsqueda.
\end{description}


\subsection*{Buscar por metadatos temporales}

\paragraph{Caso de uso \#17:}

\begin{description}[noitemsep]
        \item[Objetivo] Ingresar al Portal Web y realizar una búsqueda relacionada con cuando fue realizada la observación.
        \item[Actor] Usuario.
        \item[Necesidad] Deseable.
        \item[Prioridad] Baja.
        \item[Requerimientos Referenciados] 5.
        \item[Descripción] El usuario ingresa al Portal Web, rellena los campos de tiempo, cantidad de observaciones de un objeto y/o el intervalo entre las observaciones y realiza la búsqueda.
\end{description}

\paragraph{Caso de uso \#18:}

\begin{description}[noitemsep]
        \item[Objetivo] Ingresar al Portal Web y realizar una búsqueda relacionada con nivel de ruido, duración de la observación o tiempo de integración.
        \item[Actor] Usuario.
        \item[Necesidad] Esencial.
        \item[Prioridad] Alta.
        \item[Requerimientos Referenciados] 5.
        \item[Descripción] El usuario ingresa al Portal Web, rellena los campos de nivel de ruido, duración de la observación o tiempo de integración y realiza una búsqueda.
\end{description}


\subsection*{Buscar por polarización}

\paragraph{Caso de uso \#19:}

\begin{description}[noitemsep]
        \item[Objetivo] Ingresar al Portal Web y realizar una búsqueda por parámetros de Stokes.
        \item[Actor] Usuario.
        \item[Necesidad] Esencial.
        \item[Prioridad] Baja.
        \item[Requerimientos Referenciados] 6.
        \item[Descripción] El usuario ingresa al Portal Web, rellena los campos de parámetros de Stokes necesarios (I, Q, U y/o V) o de polarización izquierda o derecha y realiza la búsqueda.
\end{description}


\subsection*{Cruzamiento de información}

\paragraph{Caso de uso \#20:}

\begin{description}[noitemsep]
        \item[Objetivo] Buscar un objeto en múltiples fuentes de datos como archivos de los Satélites Spitzer y Herchel.
        \item[Actor] Usuario.
        \item[Necesidad] Esencial.
        \item[Prioridad] Media.
        \item[Requerimientos Referenciados] 7.
        \item[Descripción] El usuario ingresa al Portal Web, rellena los campos de nombre o código de objeto o coordenadas junto con un radio y se realiza la búsqueda. El sistema deberá realizar la búsqueda en base a ello en múltiples fuentes de información. Si se ingresa el nombre o código, se realiza una búsqueda teniendo en cuenta el margen error de cada fuente de datos.
\end{description}


\subsection*{Simulaciones}

\paragraph{Caso de uso \#21:}

\begin{description}[noitemsep]
        \item[Objetivo] Realiza una búsqueda de una simulación.
        \item[Actor] Usuario.
        \item[Necesidad] Deseable.
        \item[Prioridad] Baja.
        \item[Requerimientos Referenciados] 8.
        \item[Descripción] El usuario ingresa al Portal Web y realiza un búsqueda de una simulación.
\end{description}


\subsection*{Servicios Bibliográficos}

\paragraph{Caso de uso \#22:}

\begin{description}[noitemsep]
        \item[Objetivo] Rl realizar una búsqueda, desplegar en los resultados enlaces a información bibliográfica de SIMBAD.
        \item[Actor] Usuario.
        \item[Necesidad] Deseable.
        \item[Prioridad] Media.
        \item[Requerimientos Referenciados] 9.
        \item[Descripción] El usuario ingresa al Portal Web y al desplegarse los resultados de una búsqueda, cada resultado contiene un enlace a una búsqueda con todos los datos contenidos en SIMBAD, ADS y/o NED pertenecientes al/los objeto/s.
\end{description}

