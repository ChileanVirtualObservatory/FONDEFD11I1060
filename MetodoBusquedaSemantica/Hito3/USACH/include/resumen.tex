\section{Resumen Ejecutivo}

A partir del interés presentado por cinco universidades del país en el
ámbito de la astronomía, nace la iniciativa de crear un observatorio
virtual (CHIVO; Chilean virtual obsevatory) contando con el apoyo del
observatorio ALMA, REUNA e instituciones privadas.

ALMA (Heisig, 2007), constantemente esta generando un gran volumen de
datos, surge la necesidad entonces de desarrollar nuevas herramientas
para analizar esta información y algoritmos que sean capaces de
procesarlos.

Una de estas herramientas, que posee gran utilidad debido a sus
diversos campos de aplicación, corresponde al registro de imágenes;
técnica que se adapta dependiendo del área de aplicación y el objetivo
de estudio, aunque dicha herramienta posee gran complejidad, ha
presentado mucho interés pues existe una amplia bibliografía de
diversos autores que la avalan.

El registro de imágenes consiste en ajustar geométricamente un
conjunto de ellas, a partir de una imagen tomada como referencia, de
modo que al procesarlas coincidan espacialmente en un plano.

En el caso de imágenes astronómicas, es muy limitada su obtención. Por
un lado siempre los instrumentos de medición tendrán un margen de
error que se tiene que considerar, como también, no siempre se contará
con las mejores condiciones atmosféricas. Adicionalmente la cantidad
de factores extraterrestres que pueden influir, como por ejemplo, una
nebulosa obstaculizando el objeto de estudio, conlleva a mejorar las
técnicas de observación, y el tratamiento de la información generada.
El registro de imágenes, permite complementar la información existente
para posteriormente analizarla.

Una aplicación importante dentro del registro de imágenes, es en el
área de la medicina nuclear. Los avances tecnológicos que involucran
las imágenes médicas, han permitido pasar de una radiografía
convencional a una tomografía axial computarizada (CT), resonancia
magnética nuclear (MRI), resonancia magnética funcional (fMRI),
tomografía computarizada por emisión de fotón único (SPECT), o la
tomografía por emisión de positrones (PET). El médico actual se da
cuenta de las posibilidades que ofrece la combinación de múltiples
imágenes, pero para llevarlo a cabo y poder compararlas directamente,
es necesario eliminar las diferencias de tamaño, posicionamiento,
orientación o incluso la distorsión espacial. Todo este proceso, es el
que conoceremos como registro de imágenes. A grandes rasgos, el
registro de imágenes se puede dividir en dos etapas: primero una
transformación geométrica, con la intención de lograr la mayor
concordancia entre ellas y segundo un proceso de fusión, que es la
visualización conjunta de ellas.

Una aplicación del registro de imágenes en medicina, es el seguimiento
de la evolución de una enfermedad en pacientes epilépticos. Gracias a
esta técnica se pueden comparar estudios de un mismo paciente en
diferentes momentos, pudiendo así cuantificar diferencias entre
estudios ictales e inter-ictales con SPECT, y así localizar mejor el
foco epiléptico (Lewis, et al., 2000). Este ajuste de imágenes,
también conocido como proceso de normalización, permite comparar
resultados entre pacientes, ajustando cada paciente a una plantilla
común, de esta manera se puede realizar por ejemplo, un análisis
estadístico sobre como afecta la enfermedad, como evoluciona en
diferentes pacientes en el tiempo, o la efectividad del tratamiento.

Existen aplicaciones dentro del campo de la cartografía. Al sur de
Orinoco, en Venezuela, existe un proyecto conocido como CARTOSUR
(Miguel, 2003), el cual buscó determinar si la fusión de ortoimágenes
del radar SAR del proyecto CARTOSUR e imágenes Landsat Tematic Mapper,
permiten lograr un mejor análisis visual de los elementos del
territorio. La nubosidad propia de la región, impedía en muchos casos
hacer vuelos fotogramétricos, o que estos lograrán un área de
cobertura libre de nubes. Ambos tipos de imágenes tuvieron que pasar
por filtros para poder combinarlas, por ejemplo, las imágenes de radar
presentan un cierto tipo de ‘ruido’, conocido como moteado, el cual
produce un efecto en la imagen de granularidad. En el caso de las
imágenes de Landsat, presentan un desplazamiento en comparación a las
orto-imágenes de radar (J., 1995), las cuales posteriormente tuvieron
que ser corregidas. Finalmente se pudo concluir después de varias
pruebas, que la fusión de estos dos tipos de imágenes ayudó
definitivamente a identificar elementos de carácter antrópico y por lo
tanto, apoyar las labores de captura de datos a partir de la
interpretación de orto-imágenes de radar.

En este sentido el registro de imágenes cuenta con un conjunto de
información que permite también el estudio a nivel astronómico,
pudiendo abordar las características y variaciones de los elementos
con los que se generarán estadísticas y estimaciones en el futuro.

La solución propuesta es desarrollar un módulo en Python (Rogan \&
Muñoz, 2012), utilizando matrices y transformaciones geométricas en el
plano que normalicen imágenes en dos dimensiones. Para poder abrir,
operar y escribir sobre cada archivo, se utilizará el módulo Astropy
(Astropy) y para operaciones matemáticas generales el módulo math
(Python). De esta manera realizar un ajuste de ellas en relación a su
posición, tamaño, rotación y deproyección en un plano de dos
dimensiones, generando una colección de imágenes a analizar.