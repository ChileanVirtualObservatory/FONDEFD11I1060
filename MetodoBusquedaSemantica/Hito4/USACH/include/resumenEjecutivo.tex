\section{Resumen Ejecutivo}

El registro de imágenes consiste en ajustar geométricamente un
conjunto de ellas, a partir de una imagen tomada como referencia, de
modo que al procesarlas coincidan espacialmente en un plano.

Atacama Large Milimetric/Submilimetric Array, esta generando actualmente un gran
volumen de datos, los cuales necesitan ser analizados mediante algoritmos
eficientes.
El presente documento aborda el registro de imágenes aplicándolo a las ciencias
astronómicas en su etapa de implementación y pruebas sobre galaxias que posean 
puntualmente forma elíptica.

La solución propuesta fue desarrollar un módulo en Python (Rogan \&
Muñoz, 2012), utilizando matrices y transformaciones geométricas en el
plano que normalicen imágenes en dos dimensiones, descritas en hitos anteriores. 
Para poder abrir, operar y escribir sobre cada archivo, se utilizará el módulo Astropy
(Astropy) y para operaciones matemáticas generales el módulo math
(Python). De esta manera realizar un ajuste de ellas en relación a su
posición, tamaño, rotación y deproyección en un plano de dos
dimensiones, generando una colección de imágenes a analizar.

Para finalizar, y correspondiente a la mayor parte de este último hito, 
se realizan una serie de pruebas que miden la eficiencia,
consistencia y el rendimiento del sistema generado.


