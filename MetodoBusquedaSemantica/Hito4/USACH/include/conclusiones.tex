\section{Conclusiones}

A lo largo de este documento se ha visto como se aplica el registro de
imágenes a objetos astronómicos y la utilidad junto con el valor que
genera su uso en investigaciones en diferentes áreas mas allá de las
ciencias del universo.

Se puede apreciar la importancia que ha tenido el registro en el
ámbito de la medicina nuclear, ayudando a observar, caracterizar y
definir alteraciones físicas a nivel cerebral que presentan pacientes,
permitiendo comparar resultados en el tiempo y así definir que
tratamiento es el mas adecuado.

También se habla de un proyecto llamado CARTOSUR donde se mezclaron
imágenes en diferentes longitudes de onda. Aquí se aprecia que si bien
el registro de imágenes es una técnica que remota desde los inicios de
la imagen digital, el incursionar en nuevas técnicas, como mezclar
imágenes de radio con imágenes ópticas, es la prueba de lo amplio y
extenso que puede ser el uso del registro.

La necesidad existente en poseer mejores herramientas de análisis que
apoyen la toma de decisiones es lo que motiva la exploración,
experimentación y desarrollo de técnicas que procesen la información.

Atacama Large Milimetric/Submilimetric Array, actualmente el radio
telescopio más grande del mundo, genera tanta información, alrededor
de 750Gb de información por día, que los astrónomos requieren contar
con una plataforma abierta para acceder virtualmente a todas las
observaciones realizadas. Esta plataforma virtual nombrada ‘ChiVO’,
necesita contar con herramientas que procesen de manera inteligente
los datos capturados, generando información de calidad, útil para los
astrónomos y comunidad en general. En el caso del registro de
imágenes, es una herramienta avanzada de mucha utilidad tanto para
ALMA como otros observatorios, dando un valor agregado a la
información obtenida. Esto es de vital importancia, ya que el registro
no es tan solo disminuir el ruido o aumentar la calidad de una imagen,
sino las variadas posibilidades de análisis que permite realizar, por
ejemplo: estudiar la composición o estructura de una galaxia en
particular a partir de la información obtenida en una observación;
determinar la edad que posee un cúmulo globular; analizar la
evolución o cambios que ha tenido un cuerpo en el tiempo. En general,
los algoritmos implementados a lo largo de las etapas del registro de
imágenes son operaciones sencillas sobre matrices y transformaciones
geométricas en el plano, por lo que en la bibliografía existente no
hay mucha variación. El registro básicamente es una aplicación que
hace corresponder a cada punto $P$ de coordenadas $(x,y)$ del plano,
otro punto $P’$ de coordenadas $(x’,y’)$ del mismo plano. Si bien la
distribución de coordenadas en el plano cartesiano no poseen
exactamente la misma estructura en que se lee una matriz
computacionalmente, la geometría aplicada varía levemente, solo hay
que tener presente que lo que habitualmente leemos en un plano
cartesiano como abscisa y luego ordenada, aquí es al revés, se lee
como fila columna, y los ángulo formados en los ejes van en sentido
horario. Asumiendo lo anterior, ya es mas sencillo trabajar con la
información.

Otro aspecto fundamental en el registro de imágenes, es la forma en
que completamos la información cuando se realiza una modificación, una
matriz computacional no posee puntos densos, es decir, al subdividir
el espacio entre dos coordenadas, no necesariamente existirá otra
coordenada entera y por otro lado al trasladar la información
existente, puede que queden puntos sin información. En estos casos la
interpolación es imprescindible para completar la información
faltante. Si bien cada autor tiene su forma, la idea base es completar
a partir de los puntos vecinos el punto que esta ausente, algunos
ocupan los 4 vecinos mas cercanos, otros 16, pero depende del fin que
se quiere conseguir. En el caso de los algoritmos implementados en
este trabajo tienen una mirada distinta. Por un lado, cuando ocurren
situaciones en que la información cae sobre un punto ‘decimal’, por
ejemplo (2.4, 3), el sistema decide dejar la información repartida
entre el punto (2.0, 3) y (3.0, 3). Este esquema es mucho mas
sencillo, por un lado asegura que todos los puntos quedarán
uniformemente distribuidos y por otro lado no es invasivo, es un
trabajo mas sutil. El segundo caso ocurre cuando una imagen es
ampliada, la cantidad de información vacía intermedia que queda, es
proporcional a la razón de escala utilizada. Mirar los objetos mas
cercanos para completar el punto faltante es algo tedioso, ya que
desde un punto vacío los vecinos no están equidistantes, dada esta
situación los vecinos colaboran para completar lo que entre ellos
falta, esto es mucho mas sencillo, ya que desde un punto con
información a otro punto con información la distancia es la razón. La
cantidad de vecinos a comunicar son en grupos de cuatro, de los
cuales, dos de ellos se ocuparán para el conformar el grupo siguiente.

Un aspecto importante dentro del proceso de ajuste, es contar con el
conjunto extra de funciones que entregan la información necesaria para
que el ajuste tome valor, es decir, ya no tan solo son herramientas
que extraen, rotan, escalan, alinean o deproyectan un objeto, sino que
aportan la información necesaria para determinar en que grado conviene
rotar, en que punto alinear, que tanto agrandar o achicar, o donde
desplazar los puntos para que queden deproyectado. Con esto ya se
marca una diferencia más a una herramienta de registro convencional.

A lo largo de las pruebas realizadas para medir la eficacia de los
algoritmos, se puede decir, que fueron satisfactoriamente logradas,
pero no son lo mas óptimo al momento de operar en un computador de
escritorio. El tiempo requerido para realizar algunas operaciones como
la rotación de un objeto, registro una duración de alrededor de 20
minutos sobre un conjunto de 100 imágenes de 1095 x 1095 pixeles, sin
contar el tiempo que requeriría para terminar con el resto.

Si bien el objetivo de la tesis de realizar un módulo en Python que
realice el registro de imágenes astronómicas se cumple, es el puntapié
inicial para mejorar, optimizar y desarrollar todo lo realizado en una
plataforma virtual como es el Observatorio Virtual Chileno.
Adicionalmente se puede expandir las funcionalidades, y realizar
aplicaciones en otros tipos de objetos extrasolares, como por ejemplo,
nebulosas, discos de acreción, exoplanetas, etcétera, como también
trabajar con imágenes en otras dimensiones, por ejemplo, líneas
espectrales o cubos de información. La idea es apuntar a crear o
mejorar herramientas que potencien aún mas la información, colaborando
en el desarrollo científico y así permitir a los investigadores
invertir su tiempo en analizar o estudiar la información, mas que en
como procesarla.

Lo mas hermoso de todo lo realizado, es ver como un código toma vida y
se plasma como un sentimiento en la tarea de un astrónomo que busca
cada día descubrir y conocer lo que hay mas allá de sus ojos.