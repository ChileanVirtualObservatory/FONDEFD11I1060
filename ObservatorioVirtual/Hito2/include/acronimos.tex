\section{Indice de acrónimos}
\begin{acronym}
\acro{ChiVO}{Chilean Virtual Observatory}
\acro{VO}{Virtual Observatory}
\acro{ALMA}{Atacama Large Milimiter/submilimiter Array}
\acro{UV}{Ultravioleta}
\acro{FITS}{Flexible Image Transport System}
\acro{ASDM}{ALMA Science Data Model}
\acro{J2000}{Fecha Juliana 2451545.0 Tiempo Terrestre. Es equivalente al 1 de
enero de 2000, 11:59:27.816. Se usa para indicar un instante en el tiempo
estándar para la medición de las posiciones de los cuerpos celestes y otros
eventos estelares}
\acro{B1950}{Época besseliana, es una época basada en el Año besseliano, que es
un año tropical medido en el punto donde la longitud del Sol es exactamente
280º}
\acro{Sesame}{Es un resolvedor de nombres que retorna a partir de una cadena
que representa la designación de un objeto astronómico fuera del Sistema Solar,
la posición del objeto en el cielo y otros detalles}
\acro{SIMBAD}{Base de datos que proporciona datos básicos, identificaciones
cruzadas, bibliografía y las mediciones de los objetos astronómicos fuera del
sistema solar.}
\acro{ADS}{Astrophysics Data System, es una librería/portal digital}
\acro{NED}{SA/IPAC Extragalactic Database, ofrece datos de millones de objetos
fuera de la Vía Láctea.}
\acro{IVOA}{International Virtual Observatory Alliance}
\acro{SCS}{Simple Cone Search}
\acro{SIA}{Simple Image Access}
\acro{SSA}{Simple Spectral Access}
\acro{TAP}{Table Access Protocol}
\acro{MQ}{Message Queuing}
\end{acronym}
