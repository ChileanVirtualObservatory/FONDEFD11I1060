\section{Resumen Ejecutivo}

En el marco de la creación de una ``plataforma
astroinformática para la administración y análisis inteligente de datos a gran
escala``, se dio comienzo al desarrollo del Chilean Virtual Observatory (ChiVO).

ChiVO, en primera instancia será un portal web que le permitirá a los usuarios
acceder a datos públicos del observatorio ALMA mediante algunos protocolos de
acceso. Durante el proyecto se buscará integrar las herramientas de
procesamiento de datos astronómicos de gran escala a desarrollar por los
investigadores y equipo de trabajo del proyecto.

El desarrollo de software a medida consiste en la creación y fabricación de
sistemas informáticos que satisfacen necesidades específicas de un área. Es por
esto, que la creación de ChiVO se ha enfocado en satisfacer las necesidades que
se le presentan a la comunidad astronómica y se puso en marcha el proceso de
captura y especificación de requerimientos, los cuales han recibido
modificaciones, detalles, comentarios, de distintos expertos del área afin.

En conjunto, las universidades participantes del proyecto han mantenido
reuniones de trabajo con el fin de especificar y clarificar las funcionalidades
que esperan de un VO, siendo partícipe también el observatorio Atacama Large
Millimeter/submillimeter Array (ALMA), los cuales dispondrán los datos públicos
para que sean accesibles mediante esta plataforma. En el presente documento se
detalla:
\begin{itemize}
	\item \textbf{Casos de uso}: Casos de usos respectivos al informe de
requerimiento preliminar, identificando actores y posibles situaciones.
	\item \textbf{Arquitectura de software y diseño del servicio web}: Diseño
de arquitectura, software y bibliotecas utilizadas para la creación del
prototipo funcional del Chilean Virtual Observatory, además de consideraciones
para la escalabilidad del prototipo.
\end{itemize}

