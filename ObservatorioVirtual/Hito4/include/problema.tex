\section{Definición del Problema}

El problema abordado para este hito es la puesta en marcha
del prototipo funcional de ChiVO. Esto implica integrar los
distintos subsistemas desarrollados para el observatorio
virtual Chileno en sus versiones básicas. En el Hito 1 
se detalló el proceso de captura de requerimientos, de 
los cuales se seleccionaron los siguientes para el 
desarrollo del prototipo:

\begin{enumerate}
\item Buscar por coordenadas o región del cielo:
Se podrán realizar búsquedas de posición mediante coordenadas y radio angular
(cónicas) o por región del cielo.
\item Buscar por nombre o tipo de objeto:
El sistema deberá permitir buscar por nombres de objetos que se encuentren
definidos en Sesame, del Centre de Donnees astronomiques de Strasbourg (CDS).
\item Buscar por metadatos espectrales (frecuencia y resolución): 
Se podrán realizar búsquedas por metadatos espectrales, lo cual consiste en
búsqueda por banda o rango de frecuencia, búsquedas por líneas espectrales y 
corrimiento al rojo o búsquedas por resolución espectral.
\end{enumerate}

De estos requerimientos, se desprenden los siguientes casos 
de uso aplicables a esta entrega
(descritos en detalle en la memoria de Walter Fariña~\cite{}):
\begin{itemize}
\item CU1: Filtrar los resultados de la búsqueda en el Portal Web.
\item CU2: Descargar datos desde el Portal Web.
\item CU8: Ingresar al Portal Web y realizar una búsqueda por coordenadas.
\item CU9: Realizar una búsqueda de listado de coordenadas
\item CU10: Buscar por nombre o tipo de objeto.
\item CU13: Ingresar al Portal Web y realizar una búsqueda espectral
\end{itemize}

Cabe destacar que el conjunto de requerimientos y casos de usos completo
corresponde a la funcionalidad final que el observatorio virtual Chileno
debería ofrecer en un futuro. Por otro lado, los requerimientos y casos de usos seleccionados
para este prototipo corresponden a los desafíos tecnológicos y funcionalidades
básicas que todo observatorio virtual debiera ofrecer.

La arquitectura de ChiVO, presentada en hitos anteriores, corresponde
a un modelo de tres capas siguiendo las recomendaciones de IVOA.
%includeimage

\begin{itemize}
\item Capa Datos: modelo de datos, archivos y acceso a bajo nivel 
a los datos del observatorio virtual. Cada conjunto de datos puede
tener sus propios servicios de acceso a datos. En este prototipo
se decidió trabajar con el backend de DACHS.
\item Capa de Aplicación: web-services con única interfaz para todos
los servicios que ofrece ChiVO en su conjunto. Este endpoint está programado en
python+flask, y funciona como multiplexor de servicios. Las aplicaciones
IVOA compatibles interoperan con este endpoint.
\item Capa de Usuarios: interfaz gráfica limitada para acceder a los
a algunos servicios ofrecidos por ChiVO via un portal web. Cara visible
de ChiVO para usuarios inexpertos.
\end{itemize}

El problema para esta entrega es integrar las capas, y corroborar
los requerimientos end-to-end con datos reales. Cada caso de uso
define tareas por cada capa se tabulan a continuación.

%\footnotesize
\begin{table}[ht!]
    \begin{center}
	\begin{tabular}{c|p{1.2in}|p{1.4in}|p{1.3in}|p{1.3in}}
	    %\footnotesize
	    \textbf{CU} & \textbf{Capa Datos} &
	    \textbf{Capa de Aplicación} & \textbf{Capa de Usuarios} &
	    \textbf{Integración} \\\hline\hline
	    \#1 & 
	    Ingestar datos FITS de ALMA & Incluir servidores de otros
	    VOs & Filtrar datos desde el VOView & Implementar ciclo
	    FITS-VOTable-VOView \\\hline
	    \#2 & Desacoplar los binarios en servidor de archivos
	    independiente & Redireccionar acceso a servidor de
	    archivos & Habilitar links en VoView & Descargar archivos
	    desde portal web de forma transparente \\\hline
	    \#8 & Levantar servicios TAP, SCS y SIA & Harvesting de
	    servicios SCS y SIA & Implementar vistas definitivas de
	    SCS y SIA & Casos de prueba para datos FITS de ALMA \\\hline
	    \#9 & & & Implementar peticiones en batchs utilizando CSV &
	    Obtener resultados de queries a partir un archivo CSV
	    \\\hline 
	    \#10 &
	    Estudiar opciones de resolvedor de nombres para catálogos
	    propios & Implementar Javascript que resuelva & nombres
	    desde SESAME & Casos de pruebas utilizando nombres \\\hline
	    \#13 & & 
	    Harvesting de servicios SSA & Implementar vistas
	    definitivas de SSA & Casos de prueba para datos
	    espectrales \\
	\end{tabular}
    \end{center}
    \caption{Tareas asociadas a cada caso de uso.}
\end{table}

%\normalsize


