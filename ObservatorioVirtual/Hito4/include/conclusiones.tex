\section{Conclusiones}

Para este hito, se generaron las capacidades para la integración
de los servicios en un producto funcional. 
Actualmente, el prototipo soporta los servicios básicos
del observatorio virtual TAP, SCS, SIA y SSA, a través de las 3
capas\footnote{En rigor, el protocolo SSA no aplica a los datos de ALMA, por lo
que no se provee un servicio de SSA en la capa de datos.}.

Si bien a nivel conceptual
lo primordial es que el prototipo sea interoperable desde otros servicios VO y
utilizable con las herramientas VO que la comunidad ofrece, en este hito
se completó la mayoría de las funcionalidades requeridas del portal web, 
que es la cara visible de ChiVO. Sin embargo, aún queda por finalizar varias 
funcionalidades anexas que se han estudiado, que si bien no están en 
los requerimientos inciales (como registry, servicio de catálogos, etc.), 
son necesarias para la correcta divulgación de ChiVO a la comunidad.

Considerando lo definido inicialmente en el proyecto, se puede dar por finalizado
el desarrollo del resultado de Observatorio Virtual: ``Definir y resolver el
formato y la estructura necesaria para establecer un OV en Chile basado
en los estándarees internacionales (IVOA), lo cual reduciría la transferencia de
datos, y permitiría una mayor velocidad de acceso a los datos astronómicos.''
Para el último hito, queda la validación de estos servicios, tanto en tiempos
de respuesta como verificación del equipo astronómico.

