\section{Resumen Ejecutivo}

En el marco de esta segunda etapa de implementación del Obvervatorio
Virtual Chileno (ChiVO, \emph{Chilean Virtual Observatory}), se ha
desarrollado un prototipo funcional, en base a los requerimientos
definidos en el Hito 1 y que han sido seleccionados para esta entrega.
Los requerimientos son:

\begin{itemize}
  \item Buscar por coordenadas o región del cielo.
  \item Buscar por nombre o tipo de objeto.
  \item Buscar por metadatos espectrales.
\end{itemize}

En base a estos requerimientos se ha trabajado fundamentalmente en dos
aspectos: la capa de datos y las vistas del observatorio virtual.

En relación a la capa de datos, se ha trabajado en la ingesta de datos
(\emph{data ingestion}) de ALMA, utilizando los formatos ASDM y FITS,
para lo cual previamente se ha debido estudiar el complejo modelo de
datos de ALMA, debiendo hacer un mapeo desde éste al modelo de base de
datos propuesto por IVOA. Importante es destacar que los archivos ASDM
con los cuales se trabajó era de prueba, ya que los que posee ALMA no
son públicos, por lo que no estarán presentes dentro del prototipo
final. Algo similar se ha realizado con los datos FITS generados por
ALMA, cuya metadata se debió mapear hacia el modelo de datos propuesto
por IVOA. Finalmente, se ha debido trabajar con los archivos FITS para
poder incorporarlos a ChiVO y estudiar un mecanismo automatizado para
adquirir estos datos desde ALMA.

En la capa de aplicación, se incorporó el acceso a recursos
externos, y se depuró los servicios ofrecidos una vez 
que la funcionalidad de las capas de datos y usuarios 
se integraran. Además, se estudió las alternativas 
para la generación de un servicio de catálogos propio.

En lo que respecta a la parte visual de ChiVO, se ha implmentado las
vistas para las búsquedas \emph{Simple Cone Search}, \emph{Simple
  Image Access} y \emph{Simple Spectral Access}. Cada una de estas
búsquedas pueden ser encontradas en el sitio \url{beta.chivo.cl},
donde se encuentra el protipo funcional de la herramienta. En el menú
superior hay una pestaña llamada \emph{services}, desde la cual se
puede llegar a las 3 búsquedas. Cada una de ellas posee a su vez, 3
pestañas de características similares: la primera corresponde a un
formulario donde se debe ingresar los datos de la búsqueda; la segunda
despliega información respecto a lo ingresado en la primera,
específicamente, la lista de fuentes que contienen la información
solicitada; y la tercera despliega los resultados según las fuentes
seleccionadas en la pestaña anterior.
