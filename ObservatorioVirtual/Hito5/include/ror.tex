\section{Ruby on Rails}\label{anx:ror}

Ruby on Rails\textsuperscript{\copyright{}} es un \emph{framework} para el desarrollo web de código abierto que está optimizado para la felicidad y productividad sostenible del programador. Permite escribir un bello código favoreciendo la convención por sobre la configuración\footnote{\url{http://rubyonrails.org/}.}.

\subsection*{Ayuda para desarrollar}

Inicialmente, el ambiente de desarrollo se encontraba en la plataforma Heroku\footnote{\url{https://www.heroku.com/}.}, plataforma como servicio (PaaS) en la nube que soporte diferentes lenguajes de programación. La ventaja de contar con una PaaS, es no preocuparse de montar un ambiente de desarrollo, debido a que se desarrolla directamente en la plataforma. Para este proyecto, ello conlleva una desventaja, y es que uno de los objetivos del proyecto es contar con la información de los observatorios astronómicos ubicados en Chile, en un observatorio virtual ubicado en Chile, y los servidores de Heroku no lo están.

Es por eso que para la etapa de despliegue, se ha decidido montar una arquitectura propia en los servidores disponibles para el proyecto. Cabe destacar que Heroku pone a disposición el código desarrollado en el sistema de control de versiones \verb;Git;, por lo que se ha tomado el repositorio, se ha clonado y se ha puesto a disposición en la plataforma GitHub\footnote{\url{https://github.com/ChileanVirtualObservatory/web}.}. Para el despliegue, se ha creado una rama denominada \verb;deployment;\footnote{\url{https://github.com/ChileanVirtualObservatory/web/tree/deployment}.}.

Si se desea obtener el código con el cual se está realizando el despliegue, se debe realizar lo siguiente:

\begin{itemize}
	\item \verb;$ git clone -b deployment git@github.com:ChileanVirtualObservatory/web.git;
	\item \verb;$ cd web/\; (carpeta con el repositorio web localmente sólo con la rama \verb;deployment;).
	\item \verb;$ bundle install;
	\item \verb;$ rake db:migrate RAILS_ENV=deployment;
\end{itemize}

Para mantener actualizado el repositorio, se debe realizar lo siguiente:

\begin{itemize}
	\item Actualizar copia local: \verb;$ git pull origin deployment;
	\item Subir cambios a servidor: \verb;$ git push origin deployment;
\end{itemize}
