\section{Definición del Problema}

El problema abordado para este hito es el despliegue del sistema en una infraestructura de estándares profesionales, lo cual inicialmente no ocurrió. Como ya se ha mencionado, el sistema se trabajó en una arquitectura de 3 capas, formando a comienzos del proyecto 3 equipos de trabajo, cada uno encargado de cada capa. El problema de esto fue que cada equipo trabajó de manera independiente, utilizando versiones propias de cada software de desarrollo y no verificando la compatibilidad entre los diferentes \emph{frameworks} utilizados.

En la Tab.~\ref{tab:versiones} se detalla la lista de software utilizado con sus respectivas versiones para cada una de las capas del sistema.

\begin{table}[ht!]
	\begin{center}
		\begin{tabular}{c|c|c|c}
			Software & Backend & Endpoint & Frontend \\\hline\hline
			\texttt{python} &  &  &   \\\hline
			\texttt{python-flask} &  &  &   \\\hline
			\texttt{} &  &  &   \\\hline
			\texttt{} &  &  &   \\\hline
			\texttt{} &  &  &   \\\hline
			\texttt{} &  &  &   \\\hline
			\texttt{} &  &  &   \\\hline
			\texttt{} &  &  &   \\\hline
			\texttt{} &  &  &   \\\hline
			\texttt{} &  &  &   \\
		\end{tabular}
	\end{center}
	\caption{Listado de versiones de software utilizadas por capa.}\label{tab:versiones}
\end{table}

Idealmente, en un proyecto esto se debe preveer iniciado el mismo. No obstante, este proyecto tiene por finalidad entregar un prototipo funcional del Observatorio Virtual, por lo que la profesionalización del despliegue aún podría ser postergada. El problema es que mientras más demora la ejecución de las actividades del despliegue, más aumenta su complejidad, por lo que se considera estar en el tiempo preciso para hacerlo.

Es importante destacar que los recursos físicos actualmente disponibles son suficientes para desplegar la herramienta con una cantidad considerable de datos, por lo que se pondrá en producción el prototipo funcional que permitirá dar a conocer el sistema tanto a la comunidad astronómica, como a la comunidad en general. 

Por otro lado, en este hito se aborda la problemática de contar con una documentación apropiada para la herramienta. Si bien la interfaz gráfica de ChiVO cuenta con ayuda suficiente para poder operarlo, una de las ventajas de un observatorio virtual basado bajo los estándares de IVOA, es que puede ser utilizado mediante otras herramientas. Esto no es algo muy conocido, por lo que en este hito se potenciará la documentación en línea del sistema, con un manual descriptivo de su utilización ya sea mediante la interfaz gráfica, como mediante otras herramientas como \texttt{TOPCAT}\footnotemark.

\footnotetext{\url{http://www.star.bris.ac.uk/~mbt/topcat/sun253/index.html}}
