\section{Metodología de Trabajo}

En primer lugar se debe establecer que existe 2 grandes áreas de trabajo, una encargada de la investigación y otra encargada del desarrollo de la aplicación. En base a esta división, se ha establecido dos coordinadores por área, quienes son los encargados de establecer el contacto directo con los investigadores en el ámbito de la astronomía, computación y astroingeniería por un lado, y los desarrolladores por el otro. Ambos coordinadores a su vez, velan por la integración de los resultados de la investigación científica con el desarrollo tecnológico de la herramienta denominada Observatorio Virtual.

En el área de desarrollo, para esta etapa, se ha definido 3 equipos de trabajo: uno encargado de la ingestión de datos. Este equipo es encargado de tomar principalmente los datos del mandante (ALMA) y otros que puedan ser incluidos y poblar la base de datos de ChiVO; el segundo equipo trabajó en la integración y pruebas del sistema. En este contexto se terminó de integrar las 3 capas del sistema (\emph{backend}, \emph{endpoint} y \emph{frontend}) y se probó su funcionamiento en la infraestructura destinada para el sistema en producción; el tercer equipo se encargó de migrar todo el desarrollo a la infraestructura de producción, afinando algunos detalles que quedaban por corregir en el sitio web.

Cada uno de los equipos se reunía periódicamente para trabajar en las labores designadas, y semanalmente se realizaba una reunión con los 3 equipos para coordinar acciones en conjunto.

Por otro lado, una vez al mes se realizaba una reunión de coordinación con todos los actores relacionados al proyecto, donde se coordinaba aspectos técnicos, científicos y de administración del proyecto. En estas reuniones se coordinaba además, encuentros entre el coordinador de investigación con los investigadores de todas las universidades participantes, cuando fuere pertinente.

Toda la documentación se maneja en una plataforma en la nube para el almacenamiento de archivos, lo que permite una fácil y rápida colaboración entre los participantes de diferentes universidades del país integrantes del proyecto. Además, la codificación de la aplicación se maneja bajo un sistema de control de versiones distribuido, lo que permite el trabajo de los diferente equipos de desarrollo sin mayores problemas.

