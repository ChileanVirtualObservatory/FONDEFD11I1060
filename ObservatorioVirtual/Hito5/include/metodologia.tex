\section{Metodología de Trabajo}

En primer lugar se debe establacer que existe 2 grandes áreas de
trabajo, una encargada de la investigación y otra encargada del
desarrollo de la aplicación. En base a esta división, se ha
establecido dos coordinadores por área, quienes son los encargados de
establecer el contacto directo con los investigadores en el ámbito de
la astronomía, computación y astroingeniería por un lado, y los
desarrolladores por el otro. Ambos coordinadores a su vez, velan por
la integración de los resultados de la investigación científica con el
desarrollo tecnológico de la herramienta denominada Observatorio Virtual.

En el área de desarrollo existen 3 equipos de trabajo, cada uno de
ellos encargado de las capas de desarrollo de ChiVO y con un líder
técnico a cargo. A saber, los equipos son: \emph{backend},
\emph{endpoint} y \emph{frontend}. Semanalmente se lleva a cabo una
reunión técnica entre los coordinadores de investigación y desarrollo,
junto con los líderes de cada uno de los equipos, donde se fijan los
objetivos a cumplir y se revisa los avances alcanzados.

Por otro lado, una vez al mes se realiza una reunión de coordinación
con todos los actores relacionados al proyecto, donde se coordina
aspectos técnicos, científicos y de administración del proyecto. En
estas reuniones se coordina además, encuentros entre el coordinador de
investigación con los investigadores de todas las universidades
participantes, cuando se estime pertinente.

Finalmente, tanto el director del proyecto como el coordinador de
investigación, mantienen una constante comunicación con el mandante
del proyecto, ALMA, mostrando los avances del proyecto y como estos se
alinean con lo requerido, así como también solicitando información
(principalmente datos de prueba) para el desarrollo de ChiVO.

Toda la documentación se maneja en una plataforma en la nube para el
almacenamiento de archivos, lo que permite una fácil y rápida
colaboración entre los partipantes de diferentes universidades del
país integrantes del proyecto. Además, la codificación de la
aplicación se maneja bajo un sistema de control de versiones
distribuido, lo que permite el trabajo de los diferente equipos de
desarrollo sin mayores problemas.

%La arquitectura de deployment consiste en un servidor interno de
%almacenamiento (scruffy), uno de aplicaciones
%(dachs.lirae.cl) y un servidor para la aplicación para el portal
%web (beta.chivo.cl). Además, se utilizaron servidores clones para el desarrollo.
%Una vez que las funcionalidades se implementaron y probaron unitariamente
%en los servidores clones, se integraron en los servidores respectivos
%del prototipo. 
