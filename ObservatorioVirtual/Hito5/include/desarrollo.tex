\section{Desarrollo de la solución}

Como se ha mencionado en las secciones anteriores, este hito comprende el desarrollo de dos grandes secciones. La primera está relacionada con el despliegue de la aplicación, de tal manera de contar con una arquitectura moderna, escalable, segura y por sobre los estándares mínimos requeridos para la producción de software. El segundo tiene relación con una adecuada documentación en línea que permita una completa utilización de ChiVO, tanto bajo su interfaz web como su uso mediante otras herramientas.

Para realizar un exitoso despliegue se requiere de una infraestructura de acuerdo a las necesidades requeridas. Esto puede parecer una obviedad, pero es importante destacarlo pues no existe una fórmula única para determinar la arquitectura a utilizar. Dado esto, se realizó un completo estudio de las alternativas existentes y que estuvieran al alcance de este proyecto, para luego tomar las decisiones de qu\'e opciones tomar e implementarlas. La descripción de esto se encuentra en el Anexo \ref{anx:virtualizacion}. Además, en la sección de anexos se encuentra la descripción de cómo instalar: DaCHS (Anexo \ref{anx:dachs}), herramienta fundamental del \emph{backend}; \texttt{python-flask} (Anexo \ref{anx:flask}), \emph{miniframework} en el que fue desarrollado el \emph{endpoint}; y Ruby on Rails (Anexo \ref{anx:ror}), \emph{framework} en el que fue desarrollado el \emph{frontend}.

A continuación se detalla cada una de estas secciones y las actividades que conllevan.

\subsection{Despliegue}

La implementación o despliegue de un software corresponde a un conjunto de actividades cuyo objetivo final es hacer que el sistema de software quede listo para ser usado. Si bien existen definiciones estandarizadas de cómo realizar un despliegue de software tales como CMM \cite{cmm}, ITIL \cite{itil} y SEWBOK \cite{swebok}, las cuales proveen de directrices generales, sus descripciones son a menudo muy generales y omiten las ideas desarrolladas a partir de los retos y soluciones experimentadas por profesionales del área. Por lo mismo, se presenta en este informe una clasificación descriptiva de las diferentes actividades inmersas dentro del despliegue de software \cite{deployActivities}:

\begin{description}
	\item [Comunicación con los interesados (stakeholders)] La transferencia de conocimiento sobre el producto de software, su despliegue, y cambios en la funcionalidad interna son críticos para un despliegue exitoso. A continuación se describe las actividades que han permitido el despliegue de ChiVO sin mayores inconvenientes:
		\begin{description}
			\item [Informar a los interesados de los contenidos del despliegue] La meta de esta actividad es mantener a los interesados en el proyecto informados sobre el despliegue. Esto es de suma importancia, pues a pesar que el proyecto se encuentra aún en fases de desarrollo, se encuentra una versión beta\footnotemark{} disponible en \url{http://beta.chivo.cl/}, la cual es accedida por todos los interesados del proyecto, e incluso promocionada. Por lo mismo ha sido importante señalar que su funcionamiento puede verse afectado por mejoras implementadas, y más aún, cuando se está en una fase de despliegue. En esta fase el sistema muchas veces no estará disponible en la dirección antes mencionada, por lo que se ha informado tanto a los equipos de desarrollo, como a investigadores, profesores y director del proyecto, lo que se está realizando y el por qu\'e.

				\footnotetext{Corresponde a una fase del desarrollo de software que representa una primera versión del producto, que por lo general se presenta como software inestable y que sirve para realizar pruebas con el sistema operando.}
			\item [Entrenamiento de los usuarios] El Observatorio Virtual es una herramienta orientada fundamentalmente a la comunidad astronómica, pero tambi\'en a la comunidad en general. Por lo mismo es bastante complicado acotar el número de usuarios, e incluso identificarlos, como para entregar una capacitación personalizada. Por ello es que se ha trabajado en una documentación en línea que permita a cualquier usuario utilizar el sistema, en sus diferentes modalidades. Esto se explica con mayor detalle en la sección \S\ref{sec:doc}.
			\item [Soporte para los usuarios] De manera similar al punto anterior, se dificulta realizar esta actividad de manera personalizada, por lo que se habilita un correo electrónico\footnotemark{} para recibir las consultas, sugerencias y comentarios del sistema, y así poder tener un canal de comunicación con los usuarios con el objetivo de dar mejoras al sistema en pos de una mayor satisfacción de quienes utilicen el sistema.
				\footnotetext{\url{mailto:info@chivo.cl}.}
		\end{description}
	\item [Preparaciones para la instalación] Antes que la herramienta sea instalada, se debe procurar que todo est\'e dispuesto y de manera correcta para que la instalación sea una proceso lo más limpio posible.
		\begin{description}
			\item [Importación de los datos iniciales del cliente] Uno de las primeras actividades antes de la instalación, es la carga de datos provistos, en nuestro caso, por el mandante, ALMA. Durante largos meses se ha estado en contacto con personal de ALMA quienes nos han provisto de una cantidad importante de datos, los cuales han sido dispuestos en el ambiente de desarrollo y con ellos realizado las pruebas al sistema.
			\item [Configuración del producto] Bajo las denominaciones del proyecto, existe un mandante, que en estricto rigor corresponde al cliente del proyecto. A comienzos del proyecto se levanto una lista de requerimientos con el cliente, los cuales han sido cumplidos cabalmente. Por otro lado, existe un listado de estándares (IVOA) sobre los cuales se basa el desarrollo de este proyecto, los cuales han sido seguidos al pie de la letra. Todo esto compone la configuración del producto, dado que su utilización puede ser realizada tanto en su versión web, como tambi\'en puede ser utilizado como un servicio mediante otras aplicaciones. Al seguir los estándares, la configuración para que ChiVO funcione bajo estas herramientas ya está realizado.
			\item [Integración del producto] Similar a lo expuesto al punto anterior, la estandarización sobre la cual está desarrollado ChiVO, hace que su integración con otros observatorios virtuales y herramientas para operar sobre catálogos y tablas, tambi\'en se haya realizado exitosamente. Actualmente el observatorio virtual chileno puede ser utilizado con herramientas como \verb;TOPCAT;\footnote{\url{http://www.star.bris.ac.uk/~mbt/topcat/sun253/index.html}.}.
			\item [Planificación de una fecha para el despliegue] A fines del año 2014 se ha realizado una planificación preliminar orientada a calzar con las fechas de entrega de hitos finales y cierre del proyecto. Es así como a comienzos de año 2015 se ha ratificado 2 fechas claves: 27 de marzo y 24 de abril. La primera corresponde a un lanzamiento interno de la aplicación, en donde los integrantes del proyecto presentarán el funcionamiento de la herramienta bajo sus diferentes perspectivas. De este lanzamiento podrán surgir interrogantes y detalles a mejorar para la segunda fecha, donde \emph{the Chilean Virtual Observatory} será desplegado y lanzado para su utilización. 
			\item [Crear un paquete de despliegue] El proyecto ha contado desde sus inicios con un administrador de sistemas, quien es el encargado de toda la infraestructura tanto física como en la nube del proyecto. Una de sus labores ha sido desde siempre automatizar las instalaciones de los diferentes paquetes de software utilizados para el desarrollo y despliegue de ChiVO, documentando su uso y los pasos que no pueden ser automatizados. Es así como en la actualidad se encuentra automatizada gran parte de la instalación de ChiVO. Cabe destacar que esta herramienta se instala sólo a nivel de servidor, por lo que no acarrea gran problema aquellos pasos no automatizados. Los clientes sólo requieren de un navegador web para utilizar la herramienta, u otras herramientas que quizá deban ser instaladas a nivel de cliente, lo cual depende exclusivamente del proveedor de esa herramienta y el usuario.
		\end{description}
	\item [Instalación] En este apartado se describe las actividades necesarias para realizar la instalación de la herramienta.
		\begin{description}
			\item [Comprobación previa a la instalación] Antes de la instalación en sí, se debe realizar varias comprobaciones, fundamentalmente relacionadas con el esquema de virtualización y los sistemas operativos dispuestos en los servidores virtuales donde se instalará ChiVO, así como tambi\'en las versiones de los paquetes de software que soportarán al observatorio virtual. Una descripción detallada de todo esto se encuentra en los anexos \S\ref{anx:virtualizacion}, \S\ref{anx:dachs}, \S\ref{anx:flask} y \S\ref{anx:ror}.
			\item [Hacer posible un \emph{rollback}] Tanto el código fuente como la base de datos se maneja sobre sistemas de control de versiones. Esto permite que ante eventuales cambios que puedan mermar la funcionalidad del sistema, se pueda volver a un punto anterior cuya funcionalidad est\'e al 100\%. Es importante destacar en este punto que los cambios que pudieran hacerse al código jamás se harán ``en caliente'', esto es, en el sistema que está en producción. Los cambios a realizar siempre se harán sobre un ambiente de desarrollo. Futuras actualizaciones estarán relacionadas a mejoras y adecuaciones a nuevas versiones de los paquetes de software necesarios para su funcionamiento. No obstante aquello, siempre se mantendrá una versión estable a la cual volver en caso de alguna emergencia.
			\item [Instalación del producto] Como ya se ha mencionado anteriormente, ChiVO es una herramienta que se encuentra en la nube, por lo que su utilización puede ser de dos maneras: utilizando su interfaz web o utilizando otras herramientas del tipo \emph{standalone}. Para el primer caso, el usuario no debe instalar absolutamente nada. Sólo basta con que cuente con una navegador, el cual se da por entendido que ya tiene instalado. Si desea utilizar otra herramienta para conectarse a ChiVO, deberá leer la documentación de la herramienta en particular para poder instalarla y utilizar los servicios disponibles por ChiVO. Dado lo anterior, la herramienta debe ser instalada sólo en el lado del servidor, lo cual se ha realizado sin mayores problemas, tanto en un ambiente de desarrollo como en un ambiente de producción y con la mayoría de los pasos automatizados. Cabe destacar que futuras actualizaciones de la herramienta se llevarán a cabo en primer lugar en un ambiente de pruebas, y cuando pase las pruebas de funcionamiento, será llevado al ambiente de producción, siendo estos cambios transparentes para el usuario.
			\item [Transferir el producto desde un ambiente de pruebas a uno de producción] Al ser ChiVO una herramienta disponible en la modalidad de software como servicio (\emph{Software as a Service, SaaS}), no existe un ambiente de pruebas en el lado del cliente, si no que solamente lo hay en el lado del servidor. Durante el periodo de desarrollo de la herramienta, se puso a disposición una versión beta\footnote{\url{http://beta.chivo.cl/}.}, la cual constantemente ha sido actualizada con mejoras y arreglo de problemas. Una vez estabilizada esta versión, luego de innumerables pruebas, se ha establecido una versión estable de la herramienta a la cual se podrá acceder en la dirección \url{http://www.chivo.cl/}\footnote{Esta versión estará disponible desde el día 24 de abril de 2015, fecha del lanzamiento oficial de la herramienta.}, lo cual se ha realizado simplemente transfiriendo los archivos desde el ambiente de pruebas a una ambiente de producción. De ahora en adelante, la versión estable tambi\'en es mantenida en un ambiente de desarrollo, donde se trabajará en futuras mejoras, las cuales pasarán a una versión estable cuando corresponda.
			\item [Mantenimiento de la información sobre el producto desplegado] En el pie de página del sitio web de la herramienta, se encuentra la información sobre la versión de la misma. Además, en la sección \S\ref{sec:doc} se podrá encontrar el detalle de la documentación en línea del sistema, fundamentalmente en lo que respecta a su desarrollo.
		\end{description}
	\item [Pruebas al producto instalado] Aunque todas las versiones son probadas durante el proceso de desarrollo, ciertas actividades de prueba son todavía necesarias durante el despliegue. A continuación se describe cada una de ellas.
		\begin{description}
			\item [Pruebas por parte de los desarrolladores] Una vez alcanzada una versión estable, la herramienta ha sido probada por parte de los desarrolladores, debiendo corroborar el correcto funcionamiento según la descripción del documento de casos de uso. Los tipos de prueba realizados fueron los siguientes:
				\begin{description}
					\item[Pruebas de instalación] Asegurarse que el sistema quede instalado correctamente.
					\item[Pruebas de compatibilidad] Al ser una herramienta web, se verifica que funcione con diferentes navegadores: Google Chrome\footnote{\url{https://www.google.com/chrome/}.}, Firefox\footnote{\url{https://www.mozilla.org/es-CL/firefox/new/}.}, Iceweasel\footnote{\url{https://wiki.debian.org/Iceweasel}.}, Internet Explorer\footnote{\url{http://windows.microsoft.com/es-es/internet-explorer/download-ie}.} y Safari\footnote{\url{https://www.apple.com/cl/safari/}.}. Además, se prueba algunas herramientas que permiten conectarse a ChiVO mediante sus interfaces, como \textsc{TOPCAT}\footnote{\url{http://www.star.bris.ac.uk/~mbt/topcat/}.} y \textsc{Aladin}\footnote{\url{http://aladin.u-strasbg.fr/}.}.
					\item[Pruebas de humo] Actualmente la herramienta soporta 3 tipos de búsquedas, las cuales fueron rápidamente probadas, y así determinar si seguir o no con otras pruebas. De haber fallado, se debía corregir inmediatamente antes de seguir con las otras pruebas.
					\item[Pruebas de regresión] Iniciada la etapa de despliegue, se pudo observar algunas funcionalidades perdidas. Se tenía la certeza que \'esta había sido desarrollada, pero ya no se encontraba presente. Allí se realiza una prueba de regresión para determinar en qu\'e momento esta funcionalidad fue eliminada, y cómo puede ser recuperada.
					\item[Pruebas funcionales] Se elaboró baterías de pruebas para cada una de las búsquedas, las cuales fueron ejecutadas para verificar el funcionamiento de cada una de las características que deben estar presente en la herramienta.
					\item[Pruebas no funcionales] Con la misma batería de pruebas del punto anterior, se probó el rendimiento de la aplicación, siendo comparada con otros observatorios virtuales. Estas pruebas arrojaron un rendimiento bastante similar a sus pares extranjeros.
					\item[Pruebas destructivas] Se realizó pruebas con la finalidad que el sistema falle. Fue así como se determinó que en la lista de solicitudes que pueden ser pasadas a la vez, existe una cota máxima, la cual fue restringida con un mensaje previa a ser ejecutada, y así evitar que el sistema se caiga.
					\item[Pruebas de internacionalización y localización] Se prueba que el ingreso con caracteres en castellano no hagan fallar al sistema. En particular se realiza esto pues el sistema si bien trabaja sobre estándares que determinan que los nombres de objetos astronómicos se encuentren en ingl\'es, al ser un sitio chileno existe una probabilidad considerable que sean ingresados por error caracteres en castellano.
					\item[Pruebas A/B] Cada una de las búsquedas posee diferentes parámetros. Se realizó pruebas con cada uno de ellos y comparando las búsquedas al modificar sólo uno de ellos. Por ejemplo, en una búsqueda de datos espectrales se ingresaba los parámetros de \verb;RA;, \verb;DEC; y \verb;Size;, dejando el parámetro \verb;Band; en blanco. Luego se realizaba la misma búsqueda pero se ingresaba un valor para \verb;Band;, y se comparaba los resultados. Esto permite saber que la búsqueda con parámetros opcionales estuviera funcionando correctamente.
					\item[Pruebas de concurrencia] Con el objetivo de medir la robustez y rendimiento del sistema, se realizó pruebas continuas en condiciones normales defectos en la herramienta al haber actividad adicional. No se encontró anomalías con ejecutar pruebas concurrentes.
				\end{description}
			\item [Pruebas por parte de los usuarios] Luego de realizadas las pruebas por parte de los desarrolladores, el sistema fue presentado a potenciales usuarios (comunidad astronómica), con el fin de recibir comentarios y sugerencias por parte de ellos. Las pruebas con los usuarios fueron las siguientes:
				\begin{description}
					\item[Pruebas beta] Se selecciona un grupo acotado de usuarios del sistema, a quienes se les invita a probar la herramienta para recibir una retroalimentación. De estas pruebas se definió una versión final a ser presentada abiertamente a diferentes usuarios.
					\item[Pruebas de aceptación] El sistema es presentado a los usuarios, quienes lo utilizan en sus propios entornos y entregan comentarios de mejoras muy menores, las cuales han sido corregidas para su completa aceptación.
					\item[Pruebas de usabilidad] Se chequea con usuarios que la interfaz sea fácil de utilizar y entender. No hay reparos por parte de los usuarios.
				\end{description}
		\end{description}
\end{description}

\subsection{Documentación}\label{sec:doc}

El Observatorio Virtual Chileno fue desarrollado sobre una arquitectura de 3 capas: \emph{backend}, \emph{endpoint} y \emph{frontend}. El desarrollo de cada una de las capas ha sido debidamente documentado y puesto a disposición de los interesados en diferentes plataformas en línea, para facilitar su acceso. A continuación se detallará la ubicación de la documentación de cada una de las capas:

\begin{description}
	\item[\emph{Backend}] La capa de datos está compuesta fundamentalmente por la herramienta \textsc{DaCHS}, la cual es descrita en el anexo \S\ref{anx:dachs}. Además, los archivos binarios manejados se encuentran en una base de datos \textsc{PostgreSQL}. La documentación de esta capa puede ser encontrada en las siguientes direcciones:
		\begin{itemize}
			\item \url{http://twiki.csrg.cl/twiki/bin/view/LIRAE/WorklogDachsToolkit}
			\item \url{http://twiki.csrg.cl/twiki/bin/view/LIRAE/DaCHSResourceDescription}
			\item \url{http://twiki.csrg.cl/twiki/bin/view/LIRAE/CBEWorklogsPatricoRamirez}
			\item \url{http://twiki.csrg.cl/twiki/bin/view/LIRAE/ChivoPrototypeResourceWorklog}
		\end{itemize}
		En el listado precedente se encontrará toda la documentación desde como instalar las herramientas, hasta su funcionamiento.
	\item[\emph{Endpoint}] La capa de aplicación fue desarrollada sobre el \emph{miniframework} \verb;python-flask;, cuya descripción puede ser revisada en el anexo \S\ref{anx:flask}. Toda la documentación de lo realizado en esta capa se puede encontrar en la siguiente lista de direcciones:
		\begin{itemize}
			\item \url{https://github.com/ChileanVirtualObservatory/flask\_endpoint}
			\item \url{http://twiki.csrg.cl/twiki/bin/view/LIRAE/FranciscoLiraEndpointDocumentationWorklog}
			\item \url{http://twiki.csrg.cl/twiki/bin/view/LIRAE/EPWorklogsCamiloValenzuela}
			\item \url{http://twiki.csrg.cl/twiki/bin/view/LIRAE/EPWorklogsFranciscoLira}
		\end{itemize}
	\item[\emph{Frontend}] La capa de vistas fue desarrollado sobre el \emph{framework} \verb;Ruby on Rails;. Toda la documentación sobre esta capa se encuentra en la siguiente dirección:
		\begin{itemize}
			\item \url{http://redmine.csrg.cl/projects/front-end/wiki}
		\end{itemize}
\end{description}

Además, se ha agregado una breve documentación de funcionamiento de la herramienta, en la cual se explica cómo realizar las búsquedas, ingresando cada uno de sus parámetros. Junto con ello, se deja a disposición un listado con herramientas mediante las cuales se puede acceder a ChiVO, junto con las direcciones de la documentación disponible de cada una de las herramientas. La documentación se encontrará en la siguiente dirección:

\begin{itemize}
	\item \url{http://www.chivo.cl/documentation}
\end{itemize}

Cabe destacar que el sitio \url{http://www.chivo.cl} estará disponible desde el día 24 de abril de 2015, y que actualmente la versión final se encuentra disponible en una dirección temporal, al igual que la documentación antes mencionada. La dirección temporal correspondiente a un ambiente de pruebas es \url{http://frontend.lirae.cl/}, la cual dejará de existir a partir de la fecha de lanzamiento.
