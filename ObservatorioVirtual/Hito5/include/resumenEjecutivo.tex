\section{Resumen Ejecutivo}

El despliegue y documentación de las herramientas de Observatorio Virtual ChiVO
se realizó constituyendo un equipo de desarrollo distinto al equipo
investigador, pero manteniendo la comunicación y coordinación mediante
reuniones periódicas, tal como se ha realizado durante todo el proyecto. 
Se asignaron tareas en las áreas de ingestión de datos, integración
y testing, y desarrollo web. 

El despliegue se realizó mediante un modelo
profesional de desarrollo de software basado en actividades; 
en concreto las macro-actividades realizadas son:
\begin{enumerate}
\item \textbf{Comunicación con los stakeholders:} incluye informar a los astrónomos de
las versiones de prueba, entrenamiento en tecnologías VO y
posterior soporte para los astrónomos. Estas actividades se han realizado de
forma parcial, ya que efectivamente corresponden a la última etapa del proyecto.
\item \textbf{Preparación para la instalación:} incluye ingestión efectiva de datos
ALMA, contraste de requerimientos y estándares IVOA, integración de las capas de
la arquitectura de ChiVO, planificación de la fecha de lanzamiento de ChiVO, y
empaquetamiento del software de ChiVO para su fácil instalación.
\item \textbf{Instalación:} incluye verificación de la infraestructura para 
las tecnologías utilizadas por ChiVO, habilitar servicios de desarrollo con
control de versiones, instalación efectiva en servidores virtualizados, 
crear un ambiente de producción para lanzamiento final y mantención de
la documentación en línea de ChiVO.
\item \textbf{Pruebas al producto:} incluye pruebas de instalación,
pruebas de navegadores y herramientas VO, pruebas rápidas de los servicios
IVOA con pocas fuentes, pruebas funcionales y no funcionales, entre otras.
Además, se está trabajando en las pruebas beta, de aceptación y usabilidad
por parte del usuario.
\end{enumerate}

La documentación se realizó por el lado técnico antes y durante el desarrollo
de las herramientas, por lo que cada subgrupo de trabajo tiene su respectivo
repositorio público de documentación a nivel técnico. Además, se agregó una 
breve documentación general de la herramienta para usuarios nuevos o
ocasionales. Esta documentación es brevemente explicada y debidamente
referenciada en los anexos de este documento.

En resumen, la fase de despliegue y documentación se realizó con éxito,
ya que los estándares de desarrollo exigidos forzaban el despliegue temprano
y la documentación continua. 

