\section{Conclusiones}

Este documento evidencia la exitosa culminación del desarrollo de los servicios del Observatorio Virtual, ChiVO. Utilizando una metodología de separación de tareas entre equipos de expertos, y una etapa especial de integración y pruebas, se lograron los objetivos de incluir datos inéditos de ALMA en el formato del observatorio virtual, proveyendo servicios web para otros observatorios virtuales, y un portal web para la búsqueda de contenido.  La fase de despliegue y documentación concluyó con éxito, ya que se dispone de los servicios funcionando con datos reales, y se dispone de documentación de uso y documentación técnica ad-hoc a lo esperado.

Actualmente ChiVO se encuentra desplegado en servidores dispuestos adecuadamente en una sala de servidores de la Casa Central de la Universidad T\'ecnica Federico Santa María en Valparaíso, donde ha superado una batería de pruebas que llevan a concluir que la herramienta puede ser utilizada sin problemas por diferentes usuarios. Cabe destacar que el sistema fue desplegado de manera tal que la escabilidad no es un problema del cual se deba estar preocupado. Si bien el hardware en algún momento no dará abasto, el diseño de la arquitectura hace que sólo sea necesario agregar más hardware y no modificar en absoluto lo desarrollado. Es más, si en algún momento se piensa mover la herramienta a otro centro de datos, tampoco habría que hacer grandes modificaciones. ChiVO puede ser migrado sin problemas a un esquema de virtualización similar al que se encuentra actualmente.


