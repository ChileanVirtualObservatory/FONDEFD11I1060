\section{DaCHS}\label{anx:dachs}

GAVO\footnote{German Astrophysical Virtual Observatory: \url{http://www.g-vo.org/}.} DaCHS\footnote{Data Center Helper Suite: \url{http://docs.g-vo.org/DaCHS/}.} es una infraestructura de publicación para el observatorio virtual, que incluye componentes flexibles para la ingestión y mapeo de datos, manejo de metadata integrada con un registro de publicación y soporte para muchos estándares y protocolos VO.

\subsection*{Instalación}

Los siguientes pasos se describen para una instalación en \verb;Debian GNU/Linux;. El primer paso es agregar el repositorio. Para ello se debe modificar el siguiente archivo:

\begin{verbatim}
/etc/apt/sources.list
\end{verbatim}

\noindent y agregar las siguientes líneas:

\begin{verbatim}
deb http://vo.ari.uni-heidelberg.de/debian stable main
deb-src http://vo.ari.uni-heidelberg.de/debian stable main
\end{verbatim}

Tras esto se debe agregar las claves públicas del repositorio:

\begin{verbatim}
$ wget -qO - http://docs.g-vo.org/archive-key.asc | apt-key add -
\end{verbatim}

A continuación, se debe actualizar la lista de paquetes:

\begin{verbatim}
$ aptitude update
\end{verbatim}

Ahora viene la instalación, para ello se debe ejecutar lo siguiente:

\begin{verbatim}
$ aptitude install gavodachs-server
\end{verbatim}

Dentro del proceso de instalación se crea el grupo \verb;gavo; al cual se debe unir de la siguiente manera:

\begin{verbatim}
$ adduser ID gavo
\end{verbatim}

Tambi\'en se debe dar privilegios en la BD:

\begin{verbatim}
$ su - postgres
$ createuser -s ID
\end{verbatim}

Finalmente, se debe iniciar el servicio web:

\begin{verbatim}
gavo init
\end{verbatim}

Con esto, DaCHS queda completamente instalado, el cual puede ser revisado en \verb;localhost:8080;.

\subsection*{Conexión a base de datos externa}

Se puede iniciar el servicio utilizando una BD ya creada, para esto se debe usar lo siguiente:

\begin{verbatim}
gavo init --dsn ``host=myhost.xy port=5546 user=super password=secret dbname=wisdom''
\end{verbatim}

\subsection*{Configuración de DaCHS}

Para configurar ciertas características de DaCHS se debe crear el archivo \verb;/etc/gavo.rc;. Existen diversas opciones para configurar:

\begin{verbatim}
[general]
rootDir: /data/gavo
[web]
adminpasswd: [ADMIN_PASSWORD]
bindAddress: 
serverPort: 8080
serverURL: http://dachs.lirae.cl:8080
sitename: LIRAE DaCHS
\end{verbatim}

Posteriormente, se debe considerar lo siguiente:

\begin{itemize}
	\item Luego de configurar, se debe reiniciar el servicio: \verb;gavo serve restart;.
	\item El ejemplo anterior considera que \verb;/data/gavo; es la nueva dirección de la descripción de recursos (\verb;q.rd;) y los datos (\verb;fits;, \verb;asdm;, \verb;csv;, entre otros).
	\item El esquema de archivos es generado por la herramienta.
	\item \verb;[ADMIN_PASSWORD];: para ejecutar \verb;gavo pub -a;, el servicio tiene que volverse a cargar y para ello necesita la contraseña.
	\item Con \verb;bindAddress: [espacio]; podemos acceder desde fuera de la máquina.
\end{itemize}

\subsubsection*{Configuraciones}

\begin{itemize}
	\item \verb;[web];:
		\begin{itemize}
			\item \verb;adminpasswd;: contraseña para la administración del servicio, necesaria para el comando: \verb;gavo serve reload;.
			\item \verb;adsMirror;: repositorio para ADS\footnote{Astrophysics Data System.}. Por defecto: \url{http://ads.g-vo.org/}.
			\item \verb;bindAddress;: por defecto direccionada a  '127.0.0.1'.
			\item \verb;enableTests;: paginas de prueba. Valores: \verb;True;, \verb;False;. Valor por defecto es \verb;false;.
			\item \verb;favicon;: dirección al \verb;favicon;, según \verb;webDir;.
			\item \verb;graphicMimes;: \emph{MIME types} considerados como gráficos. Valor por defecto: \verb;image/fits,image/jpeg;.
			\item \verb;maxPreviewWidth;: límite de ancho de las vistas previas, Valor por defecto: 300.
			\item \verb;maxUploadSize;: tamaño máximo de archivos subidos. Valor por defecto: 20000000.
			\item \verb;nevowRoot;: dirección de la raíz del servidor. Valor por defecto: \verb;/;.
			\item \verb;preloadRDs;: lista de RD\footnote{Resource Descriptor.} que se pre cargan al iniciar el servidor. Es una lista de \emph{strings} y su valor por defecto es: \verb;'';.
			\item \verb;previewCache;: dirección relativa a \verb;webDir;, para almacenar \emph{cache} de la vista previa. Valor por defecto: \verb;previewcache;.
			\item \verb;realm;: autenticación de dominio. Valor por defecto: \verb;X.Unconfigured;.
			\item \verb;serverPort;: puerto del servidor. Valor por defecto: 8080.
			\item \verb;serverURL;: URL relativa cuando es necesario. Valor por defecto: \url{http://localhost:8080}.
			\item \verb;sitename;: nombre del sitio, Valor por defecto: \emph{GAVO data center}.
			\item \verb;sqlTimeout;: tiempo de espera para consultas. Valor por defecto: 15.
			\item \verb;templateDir;: dirección de \verb;webDir;. Valor por defecto: \verb;templates;.
			\item \verb;user;: usuario que corre el servidor. Valor por defecto: \verb;gavo;.
		\end{itemize}
	\item \verb;[ui];:
		\begin{itemize}
			\item \verb;outputEncoding;: codificación para los mensajes del sistema. Valor por defecto: \verb;iso-8859-1;.
		\end{itemize}
	\item \verb;[ivoa];:
		\begin{itemize}
			\item \verb;authority;: ID de autorización. Valor por defecto: \verb;x-unregistred;.
			\item \verb;dalDefaultLimit;: límite exacto por defecto sobre las consultas SCS/SSAP/SIAP: 10000.
			\item \verb;dalHardLimit;: límite exacto e irrevocable sobre las consultas SCS/SSAP/SIAP: 1000000.
			\item \verb;oaipmhPageSize;: número de registros por página en la interfaz OAI-PMH. Valor por defecto: 500.
			\item \verb;votDefaultEncoding; codificación por defecto para los \verb;VOTables;: \verb;binary;.
		\end{itemize}
	\item \verb;[adql];:
		\begin{itemize}
			\item \verb;webDefaultLimit;: límite exacto por defecto para consultas ADQL a trav\'es de un formulario web. Valor: 2000.
		\end{itemize}
	\item \verb;[async];:
		\begin{itemize}
			\item \verb;csvDialect;: el formato CSV es definido por el módulo CSV de \verb;python; usado cuando se escribe el CSV. Valor por defecto: \verb;excel;.
			\item \verb;defaultExecTime;: tiempo de espera por defecto para los trabajos UWS, en segundos. Valor: 3600.
			\item \verb;defaultExecTimeSync;: tiempo de espera por defecto para los trabajos UWS sincrónicos. Valor: 60.
			\item \verb;defaultLifetime;: tiempo por defecto (en segundos) para destruir los trabajos UWS. Valor: 172800.
			\item \verb;defaultMAXREC;: límite exacto por defecto para las consultas ADQL vía UWS/TAP. Valor: 2000.
			\item \verb;hardMAXREC;: límite exacto e irrevocable para las consultas ADQL vía UWS/TAP. Valor: 20000000.
			\item \verb;maxTAPRunning;: máximo número de trabajos TAP corriendo a la vez. Valor por defecto: 2.
		\end{itemize}
	\item \verb;[db];:
		\begin{itemize}
			\item \verb;adqlProfiles;: nombres de perfiles que tienen acceso a las tablas abiertas por ADQL. Valor por defecto: \verb;untrustedquery;.
			\item \verb;defaultLimit;: límite exacto por defecto para las consultas a la base de datos. Valor: 100.
			\item \verb;interface;: no cambiar. Valor por defecto: \verb;psycopg2;.
			\item \verb;maintainers;: nombre de los perfiles que deberían tener acceso completo para crear tablas por defecto en \verb;gavoimp;. Valor: \verb;admin;.
			\item \verb;msgEncoding;: codificación de los mensajes provenientes de la base de datos. Valor por defecto: \verb;utf-8;.
			\item \verb;profilePath;: ruta para localizar los perfiles de la base de datos. Valor por defecto: \verb;~/.gavo:$configDir;.
			\item \verb;queryProfiles;: nombre de los perfiles que deberían ser capaces de leer por defecto las tablas creadas en \verb;gavoimp;.
		\end{itemize}
\end{itemize}

\subsubsection*{Cambiar de carpeta}

Si se desea cambiar de ubicación los archivos y recursos, sólo basta cambiar las configuraciones en el siguiente archivo:

\begin{verbatim}
/etc/gavo.rc
\end{verbatim}

\noindent y modificar la siguiente línea:

\begin{verbatim}
[general]
rootDir: PATH
\end{verbatim}

\noindent donde \verb;PATH; es la nueva ruta que contiene los recursos y datos. Luego es necesario dar los permisos correspondientes:

\begin{verbatim}
$ chown gavo:gavo PATH
$ chown ID:gavo -R PATH
\end{verbatim}

Finalmente, se debe reiniciar el servicio:

\begin{verbatim}
gavo serve reload
\end{verbatim}

Mientras la nueva carpeta contenga todos los recursos (\verb;*.rd;, \verb;*.fits;, entre otros) no va existir problema.

\subsection*{Control de DaCHS}

\begin{description}
	\item[Iniciar el servicio:] \verb;gavo serve start;
	\item[Reiniciar las configuraciones del servicio\footnote{Esta acción no recarga las tablas.}:] \verb;gavo serve reload;
	\item[Detener el servicio:] \verb;gavo serve stop;
	\item[Reiniciar el servicio:] \verb;gavo serve restart;
	\item[Implementar las tablas y recursos:] \verb;gavo imp resource_folder/q;
	\item[Eliminar recurso:] \verb;gavo drop resource_folder/q;
	\item[Evaluar recurso\footnote{De preferencia utilizarlo antes de implementar.}:] \verb;gavo val resource_folder/q;
\end{description}

\subsection*{Recursos predeterminados}

Si se desea importar tablas determinadas, se puede hacer de la siguiente manera:

\begin{itemize}
	\item \verb;gavo imp --system //dc_tables;
	\item \verb;gavo imp --system //services;
	\item \verb;gavo imp --system //users;
	\item \verb;gavo imp --system //adql;
	\item \verb;gavo imp --system //tap;
	\item \verb;gavo imp --system //products;
	\item \verb;gavo imp --system //obscore;
\end{itemize}

Para ver los RD de estas tablas, se puede acceder al repositorio de la fuente (\verb;svn co; \url{http://svn.ari.uni-heidelberg.de/svn/gavo/python/trunk/}), y en \verb;gavo/resources/inputs;, se encontrarán los archivos. Estos archivos se podrán usar como \verb;Mixin; para construir tablas futuras.

\subsection*{Descripción de recursos}

Se debe crear un directorio en \verb;inputs;, de preferencia con el nombre del recurso (\verb;/var/gavo/inputs/MyResource/q.rd);. Para describir el nuevo recurso, el esquema, las tablas y su metadata, entre otros, se utiliza un archivo \verb;XML; con extensión \verb;.rd;. Una vez completo el \verb;q.rd;, se debe ir al directorio donde se encuentra el archivo, y se debe ejecutar lo siguiente:

\begin{verbatim}
gavo imp q
\end{verbatim}

La construcción del archivo \verb;q.rd; se realiza de la siguiente manera:

\begin{verbatim}
<?xml version=``1.0'' encoding=``iso-8859-1''?>

<resource schema=``carpeta''>
  <meta name=``title''>Titulo</meta>
  <meta name=``creationDate''>2009-06-02T08:42:00Z</meta>
  <meta name=``description'' format=``plain''>
    Descripción.
  </meta>
  <meta name=``copyright''>Free to use.</meta>
  <meta name=``creator.name''>Author, S.; Other, A.</meta>
		 

  <!-- Acá van las etiquetas relacionados al esquema -->
  <meta name=``subject''>Large Magellanic Cloud</meta>
  <meta name=``subject''>Interstellar medium, nebulae</meta>
  <meta name=``subject''>Extinction</meta>

  <meta name=``coverage''>
    <meta name=``waveband''>Optical</meta>
  </meta>

  <meta name=``content.type''>Catalog</meta>

  <!-- Acá se define la tabla, columna y sus especificaciones --> 
  <table id=``NOMBRE_TABLA'' onDisk=``True'' adql=``True''>
    <meta name=``description''>
      DESCRIPCIÓN
    </meta>
    <column name=``NAME''
      description=``DESCRIPCIÓN''/>
    <column name=``centerAlpha''
      required=``True''
      ucd=``pos.eq.ra;meta.main''
      unit=``deg''
      tablehead=``RA''
      description=``EJEMPLO: Area center RA ICRS''
      verbLevel=``1''/>
  </table>
								     
  <!-- Especificaciones sobre la data y como manejarla -->
  <data id=``import_content''>

    <!-- Se especifica el source -->
    <sources pattern=``data/*.txt''/>
    <reGrammar topIgnoredLines=``1''>
      <names>raMin, raMax, decMin, decMax, ev_i, a_v, a_i</names>
    </reGrammar>
						    
    <!-- Información de la tabla, generada de datos ingresados a la tabla -->
    <make table=``NOMBRE_TABLA''>
      <rowmaker id=``build_exts'' idmaps=``*''>
        <var name=``raMin''>float(@raMin)</var>
        <var name=``raMax''>float(@raMax)</var>
        <var name=``decMin''>float(@decMin)</var>
        <var name=``decMax''>float(@decMax)</var>
					  
        <map dest=``centerAlpha''>(@raMin+@raMax)/2.</map>
        <map dest=``centerDelta''>(@decMin+@decMax)/2.</map>
        <map dest=``bbox''>coords.Box\left( @raMin, @decMin),
          (@raMax, @decMax))
        </map>
      </rowmaker>
    </make>
  </data>
</resource>
\end{verbatim}

\subsection*{Importar RD}

En primer lugar, se debe ir al directorio donde se encuentra el archivo \verb;q.rd;, y validarlo:

\begin{verbatim}
gavo val name_schema/q
\end{verbatim}

Con lo anterior se verificará si existe errores en la estructura del archivo. De haberlos, se deberá corregirlos, y validarlo nuevamente hasta que no exista errores. Ya sin errores, el archivo puede ser importado:

\begin{verbatim}
gavo imp name_schema/q
\end{verbatim}

Luego de lo anterior, aparecerán las tablas en el \emph{web service}.

\subsection*{Publicar servicio}

\subsubsection*{Simple Cone Search}

Para utilizar los servicios de Simple Cone Search (SCS) de manera simplificada, se debe agregar lo siguiente:

\begin{verbatim}
<service id=``SERVICE_ID'' allowed=``scs.xml,form,static''>
  <meta name=``shortName''>lmcextinct ConeSearch</meta>
  <meta name=``testQuery''>
    <meta name=``ra''>9.4076</meta>
    <meta name=``dec''>9.6414</meta>
  </meta>
  <dbCore queriedTable=``TABLE_NAME''>
    <condDesc original=``//scs#humanInput''/>
    <condDesc original=``//scs#protoInput''/>
  </dbCore>
  <publish render=``scs.xml'' sets=``local''/>
  <publish render=``form'' sets=``local''/>
  <outputTable verbLevel=``20''/>
</service>
\end{verbatim}

Algunas consideraciones:

\begin{itemize}
	\item Note que, en \verb;queriedTable=``TABLE_NAME'';, debe ir el mismo nombre de la tabla creado en este ejemplo como \verb;NOMBRE_TABLA;.
	\item Para poder acceder a los métodos de acceso, debemos elegir \verb;sets=local;, \verb;ivo_managed; los registra en el VO.
	\item Es necesario tener dos columnas con los UCD como index: \verb:pos.eq.ra;meta.main: y \verb:pos.eq.dec;meta.main:.
\end{itemize}

\subsubsection*{Simple Image Access Protocol}

Para implementar SIAP se debe ejecutar lo siguiente:

\begin{verbatim}
$ gavo imp -m RECURSO/q && gavo imp //obscore create
\end{verbatim}

con \verb;-m; para modificar la tabla y \verb;//obscore; para crear la tabla/instancia en la tabla \verb;ivoa.ObsCore;. Importante es revisar los permisos para \verb;gavo; sobre los archivos \verb;FITS;.

\paragraph{}
\subparagraph{Mixin}

El \verb;mixin //siap#pgs; agrega funcionalidades al protocolo SIA y asegura un cierto grupo de tablas para el protocolo. El \verb;mixin; se puede especificar como un hijo:

\begin{verbatim}
<table id=``table_id'' onDisk=``True'' adql=``True''>
  <mixin>
    //siap#pgs
  </mixin>
</table>
\end{verbatim}

\noindent o como un atributo:

\begin{verbatim}
<table id=``table_id'' onDisk=``True'' adql=``True'' mixin=``//siap#pgs''>
  <!--COLUMNS-->
</table
\end{verbatim}

\subparagraph{Recursos}

Dentro de la etiqueta \verb;data;, se agrega lo siguiente para indicar de donde obtener los archivos de datos:

\begin{verbatim}
<sources recurse=``True''>
  <pattern> PATH_TO_FILES/*.fits
</sources>
\end{verbatim}

\subparagraph{Grammar}

Con \verb;fitsProdGrammar\; podemos obtener el metadato del encabezado del archivo \verb;.fits;. Debemos agregar un \verb;rowfilter;, que agregará las columnas, \verb;owner; (\verb;key; para los dueños) y \verb;embargo; (Fecha de publicación):

\begin{verbatim}
<fitsProdGrammar qnd=``True''>
  <!-- MAPEO FITS HEADER -->
  <rowfilter procDef=``__system__/products#define''>
    <bind key=``table''>``siapobsexample.spe''</bind>
  </rowfilter>
</fitsProdGrammar>
<make table=``spe'' rowmaker=``ID_ROWMAKER'' />
\end{verbatim}

\verb;//products#define;, agrega a la tabla los siguientes elementos mostrados en la Tab.~\ref{tab:pd}:

\begin{table}[ht!]
	\begin{center}
		\begin{tabular}{c|c|c}
			Fila & Descripción & Valor por defecto \\\hline\hline
			\verb;accref;	& acceso de referencia (input)	& \verb;inputRelativePath{False}; \\\hline
			\verb;embargo;	& Fecha de publicación	        & None                            \\\hline
			\verb;fsize;	& tamaño input	                & \verb;inputSize;                \\\hline
			\verb;mime;	& Mime-Type	                & \verb;image/fits;               \\\hline
			\verb;owner;	& Propietario	                & None                            \\\hline
			\verb;path;	& Path al producto	        & \verb;inputRelativePath{True};  \\
		\end{tabular}
	\end{center}
	\caption{Elementos agregados por \texttt{products\#define}.}\label{tab:pd}
\end{table}

\subparagraph{rowmaker}

\begin{verbatim}
<rowmaker id=``ID_ROWMAKER''>
  <apply procDef=``//siap#computePGS''/>
  <apply procDef=``//siap#setMeta''>
    <bind name=``instrument''> </bind>
    <bind name=``title''> ``TDT''</bind>
  </apply>
</rowmaker>
\end{verbatim}

Algunas consideraciones:

\begin{itemize}
	\item \verb;//siap#setMetaM; y \verb;bind;: para almacenar claves (\verb;key;) importantes para el usuario.
		\begin{itemize}
			\item \verb;vars[``CAMPO''];: se lee el \verb;CAMPO; de fits y se evalúa. Ej.: \verb;vars[``imageTitle''];, \verb;vars[``TELESCOP''];.
			\item Con la etiqueta se puede crear campos especializados. Ej.: \verb;%s %s``%(@TELESCOP,@DATE_OBS);.
			\item Si algún campo del archivo \verb;.fits; contiene un carácter \verb;-;, debe ser reemplazado por \verb;_;.
		\end{itemize}
	\item No se debe agregar \verb;idmaps=''*``; a \verb;rowmaker;.
	\item Se puede agregar mapeo de datos desde los FITS:
		\begin{itemize}
			\item Los campos se leen de la forma \verb;@campo;.
		\end{itemize}
\end{itemize}

\subparagraph{\texttt{Service} y \texttt{dbCore}}

\begin{itemize}
	\item Se debe indicar ciertos atributos como:
		\begin{itemize}
			\item id del servicio.
			\item Servicios aprobados: Atributo \verb;allowed; en \verb;service;, indica qu\'e servicios serán publicados o mostrados.
		\end{itemize}
	\item Indica qu\'e  servicios se publicarán:
		\begin{itemize}
			\item Atributo \verb;sets;: lista de dónde serán publicados los servicios (\verb;local; o \verb;ivo_managed;).
			\item Atributo \verb;render;: indica qu\'e servicio se publica.
		\end{itemize}
\end{itemize}

\begin{verbatim}
<service id=``SIAP_NAME_ID'' allowed=``form,siap.xml''>
  <meta name=``shortName''>SIAP_NAME</meta>
  <meta name=``title''>``SIAP_TITLE''</meta>
  <publish render=``siap.xml'' sets=``local''/>
  <publish render=``form'' sets=``local'' />
  <!-- CORES -->
  <dbCore id=``ID_CORE'' queriedTable=``TABLE_NAME''>
    <condDesc original=``//siap#protoInput''/>
    <condDesc original=``//siap#humanInput''/>
  </dbCore>
</service>
\end{verbatim}

\subsubsection*{Table Access Protocol}

Dentro de la etiqueta \verb;data;, se registra el servicio TAP:

\begin{verbatim}
<data>
  \ldots
  <register services=``__system__{/tap#run''/>
</data>
\end{verbatim}

\subsubsection*{Publicar}

Se agrega \verb;adminpasswd; en \verb;/etc/gavo.rc;, y luego de tener listo el archivo \verb;q.rd;, se debe publicar los servicios:

\begin{itemize}
	\item Publicar todos los servicios: \verb;gavo pub -a;.
	\item Publicar un \verb;rd; en específico: \verb;gavo pub resource_folder/rd_name;.
\end{itemize}

Es recomendable evaluar \verb;gavo val resource_name/rd_name; antes de importar y publicar.

\subsection*{Plantilla}

El servicio web viene con una plantilla (\emph{template}) por defecto, pero se puede cambiar. Para esto se debe copiar el directorio \verb;gavo/resources/templates/; al directorio \verb;/var/gavo/web;. Editar \verb;root.html\; para cambiar la pagina principal.

\subsection*{Defaultmeta.txt}

Se puede definir metadata del \verb;publisher;, \verb;contact;, \verb;authority;, entre otros, de DaCHS. Esto es necesario configurar para \verb;registry;:

\begin{itemize}
	\item \verb;publisher: Fill Out;
	\item \verb;publisherID: ivo://x-unregistred;
	\item \verb;contact.name: Fill Out;
	\item \verb;contact.address: Ordinary street address.;
	\item \verb;contact.email: Your email address;
	\item \verb;contact.telephone: Delete this line if you don't want to give it;
	\item \verb;creator.name: Could be same as contact.name;
	\item \verb;creator.logo: a URL pointing to a small png;
\end{itemize}

