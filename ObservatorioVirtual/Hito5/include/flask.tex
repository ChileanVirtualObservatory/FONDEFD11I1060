\section{\texttt{python-flask}}\label{anx:flask}

Flask es un \emph{microframework} para Python basado en \emph{Werkzeug}, \emph{Jinja 2} y buenas intenciones. Además, está licenciado bajo 3 cláusulas de la licencia BSD \cite{flask}.

\paragraph{Requerimientos}

\subparagraph{Python}

\begin{itemize}
	\item Python $>=$ 2.7.
	\item Python Flask\footnote{\url{http://flask.pocoo.org/}}.
	\item Python requests\footnote{\url{http://docs.python-requests.org/en/latest/}}.
	\item Python urllib.
	\item Python urllib2.
	\item Python json.
\end{itemize}

\subparagraph{Despliegue}

\begin{itemize}
	\item Apache web Server\footnote{\url{http://httpd.apache.org/}}.
	\item \verb;wsgi_mod;\footnote{\url{http://flask.pocoo.org/docs/0.10/deploying/mod\_wsgi/}}.
\end{itemize}

\paragraph{Desarrollo y pruebas}

\begin{itemize}
	\item Clonar repositorio:
		\begin{verbatim}
		git clone git@github.com:ChileanVirtualObservatory/flask_endpoint.git
	\end{verbatim}
	\item Abrir un terminal y ejecutar \verb;endpoint.py;:
		\begin{verbatim}
		python endpoint.py
	\end{verbatim}
	\item El servidor web estará corriendo en \verb;localhost:5000;.
\end{itemize}

\paragraph{Despliegue}

\begin{itemize}
	\item Clonar repositorio:
		\begin{verbatim}
		git clone git@github.com:ChileanVirtualObservatory/flask_endpoint.git
	\end{verbatim}
	\item Editar \verb;endpoint.wsgi; con la siguiente ruta:
		\begin{verbatim}
		sys.path.insert(0,``/var/www/flask_endpoint'')
	\end{verbatim}
	\item Configurar Apache\footnote{\url{http://flask.pocoo.org/docs/0.10/deploying/mod\_wsgi/\#configuring-apache}}.
\end{itemize}

\paragraph{Servicios}

\subparagraph{Lista de todos los catálogos}

\begin{itemize}
	\item WEB: \url{}.
	\item M\'etodo: \verb;GET;.
\end{itemize}

\subparagraph{\emph{Table Access Protocol}}

\begin{itemize}
	\item Parámetros: \S\ref{anx:dachsparam}.
\end{itemize}

\subparagraph{Capacidades de servicio}

\begin{itemize}
	\item WEB: \url{}.
	\item M\'etodo: \verb;GET;.
\end{itemize}

\subparagraph{Consulta sincrónica}

\begin{itemize}
	\item WEB: \url{}.
	\item M\'etodo: \verb;POST;.
	\item Retorno: \verb;VOTABLE;.
\end{itemize}

\subparagraph{Consulta asincrónica}

\begin{itemize}
	\item Nueva consulta:
		\begin{itemize}
			\item WEB: \url{}.
			\item M\'etodo: \verb;POST;.
			\item Retorno: \verb;VOTABLE; con un \verb;id; de la consulta.
		\end{itemize}
	\item Lista de consultas:
		\begin{itemize}
			\item WEB: \url{}.
			\item M\'etodo: \verb;GET;.
			\item Parámetros: no hay.
			\item Retorno: \verb;UWL;, un archivo \verb;xml; con todas las consultas y sus estados.
		\end{itemize}
	\item Información de la consulta:
		\begin{itemize}
			\item WEB: \url{}.
			\item M\'etodo: \verb;GET;.
			\item Parámetros: no hay.
			\item Retorno: \verb;UWL; con es estado de la consulta.
		\end{itemize}
	\item Otras opciones:
		\begin{itemize}
			\item WEB: \url{}.
			\item M\'etodo: \verb;GET;.
			\item Parámetros: \verb;option phase;, \verb;quote;, \verb;executionduration;, \verb;destruction;, \verb;error;, \verb;parameters;, \verb;results;, \verb;owner;.
		\end{itemize}
\end{itemize}

\subparagraph{Simple Cone Search}

\begin{itemize}
	\item WEB: \url{}.
	\item M\'etodo: \verb;GET;.
	\item Atributos: 
		\begin{itemize}
			\item RA: grado decimal.
			\item DEC: grado decimal.
			\item SR: arco minuto.
		\end{itemize}
\end{itemize}

\subparagraph{Simple Image Access}

\begin{itemize}
	\item WEB: \url{}.
	\item M\'etodo: \verb;GET;.
	\item Atributos: 
		\begin{itemize}
			\item POS: grado decimal.
			\item SIZE: grado decimal.
		\end{itemize}
	\item Atributos opcionales:
		\begin{itemize}
			\item INTERSECT: \verb;COVERS;, \verb;ENCLOSED;, \verb;CENTER;, \verb;OVERLAPS;.
			\item NAXIS: \verb;pixels;.
			\item CFRAME: \verb;ICRS;, \verb;FK5;, \verb;FK4;, \verb;ECL;, \verb;GAL;, \verb;SGAL;.
			\item EQUINOX.
			\item CRPIX.
			\item CRVAL.
			\item CDELT.
			\item ROTANG: grados decimales.
			\item PROJ: \verb;TAN;, \verb;SIN;, \verb;ARC;.
			\item FORMAT: \verb;image/fits;, \verb;image/png;, \verb;image/jpeg;, \verb;text/html;, \verb;ALL;, \verb;GRAPHIC;, \verb;GRAPHIC-<SOMETHING>;, \verb;METADATA;.
			\item VERB: 1, 2, 3.
		\end{itemize}
\end{itemize}

\subparagraph{External} 

\begin{itemize}
	\item {\bf Obtener servicios JSON externos.} Para obtener un servicio específico (no todos los datos, sólo \verb;shortname;, \verb;title; y \verb;accessURL;: \url{}. Servicio puede ser \verb;tap;, \verb;scs;, \verb;sia; y \verb;ssa;.
\end{itemize}

\subparagraph{\emph{Harvesting Interface}}

\begin{itemize}
	\item Registry harvesting interface: \url{}.
\end{itemize}
